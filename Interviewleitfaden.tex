% !TEX root = main.tex
% !TEX TS-program = pdflatex
% !BIB program = biber


\lstset{language=C}
\section*{Leitfaden für das Experteninterview mit Softwarearchitekten}

\renewcommand{\arraystretch}{1.3} % Zeilenabstand in Tabelle
\setlength{\tabcolsep}{6pt}       % Spaltenabstand

\begin{tabularx}{\textwidth}{|>{\raggedright\arraybackslash}p{4.5cm}|X|}
\hline
\textbf{Thema} & Anforderungen und Herausforderungen bei der Nutzung von Modellierungssprachen durch Softwarearchitekten \\
\hline
\textbf{Ziel des Interviews} & Erhebung qualitativer Daten zur Beantwortung folgender Forschungsfragen: \newline
1. Welche Anforderungen haben Softwarearchitekten an Modellierungssprachen? \newline
2. Welche Herausforderungen und Verbesserungspotenziale sehen Softwarearchitekten bei der Anwendung bestehender Modellierungssprachen? \\
\hline
\textbf{1. Einführung (ca. 5 Min.)} & Vorstellung des Interviewers und des Projekts \newline
Ziel und Zweck des Interviews erläutern \newline
Hinweis auf Anonymität und Vertraulichkeit \newline
Zustimmung zur Aufzeichnung (falls notwendig) \newline
Kurzvorstellung des Interviewten: Name, Position, Berufserfahrung \newline
Tätigkeitsbereich (z. B. Branche, Art der Projekte) \\
\hline
\textbf{2. Allgemeiner Hintergrund (ca. 5--10 Min.)} & Wie sieht Ihre Rolle in der Softwarearchitektur typischerweise aus? \newline
Welche Modellierungssprachen verwenden Sie in Ihrer Arbeit (z. B. UML, SysML, BPMN)? \newline
Ab wann beginnt für Sie Modellierung? \\
\hline
\textbf{3. Anforderungen an Modellierungssprachen (Forschungsfrage 1, ca. 20 Min.)} & Welche Eigenschaften oder Merkmale sind Ihnen bei Modellierungssprachen besonders wichtig? (z. B. Ausdrucksstärke, Verständlichkeit, Toolunterstützung) \newline
Welche Anforderungen ergeben sich speziell aus Ihrer Praxis oder Branche? \newline
Gibt es Unterschiede in den Anforderungen je nach Zielgruppe der Modelle (z. B. Entwickler, Stakeholder, Fachbereich, Kunden)? \newline
\textit{Wie wichtig sind Formale Anforderungen?} (Korrektheit, Vollständigkeit, Redundanzfreiheit, Wiederverwendbarkeit, Wartbarkeit, Abstraktion) \newline
\textit{Wie wichtig sind Anwenderbezogene Anforderungen?} (Einfachheit, Verständlichkeit) \newline
\textit{Wie wichtig sind Anwendungsbezogene Anforderungen?} (Mächtigkeit, Angemessenheit, Operationalisierbarkeit) \\
\hline
\textbf{4. Herausforderungen und Verbesserungspotenziale (Forschungsfrage 2, ca. 20 Min.)} & Welche typischen Herausforderungen begegnen Ihnen bei der Nutzung bestehender Modellierungssprachen? \newline
Gibt es konkrete Beispiele für Missverständnisse, Fehlinterpretationen oder Kommunikationsprobleme? \newline
\textit{Welche Verbesserungspotenziale sehen Sie in Bezug auf die Semantik der Sprachen?} \newline
\textit{Welche Verbesserungspotenziale sehen Sie in Bezug auf die grafische Notation?} \newline
\textit{Welche Verbesserungspotenziale sehen Sie in Bezug auf die Toolunterstützung (z. B. Usability, Performance, Kompatibilität)?} \newline
Gibt es spezifische Anwendungsfälle, in denen bestehende Sprachen versagen oder nicht ausreichen? \newline
Haben Sie Verbesserungsvorschläge oder Ideen für neue Ansätze in der Modellierung? \\
\hline
\textbf{5. Abschlussfragen (ca. 5 Min.)} & Gibt es noch etwas, das Sie zu dem Thema anmerken möchten? \newline
Welche Literatur, Tools oder Best Practices würden Sie in diesem Kontext empfehlen? \newline
Dürfen wir Sie bei Rückfragen erneut kontaktieren? \\
\hline
\end{tabularx}