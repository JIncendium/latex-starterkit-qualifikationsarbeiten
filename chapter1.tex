% !TEX root = main.tex
% !TEX TS-program = pdflatex
% !BIB program = biber


\lstset{language=C}

\chapter{Einleitung}







\section{Problemstellung und Zielsetzung}
\label{sec:semant-textauszeichnung}

In dieser empirischen Studie sollen die Anforderungen von Softwarearchitekten an Modellierungssprachen erforscht werden. Die Arbeit wird Interviews und Umfragen mit praktizierenden Softwarearchitekten umfassen, um ein tiefes Verständnis ihrer Bedürfnisse, Präferenzen und der Herausforderungen zu gewinnen. Ziel ist es, ein klares Bild der Kriterien zu entwickeln, die Softwarearchitekten bei der Auswahl und Nutzung von Modellierungssprachen anwenden. Die Ergebnisse dieser Studie liefern praxisnahe Impulse für die Entwicklung und Verbesserung von Modellierungswerkzeugen, die adäquater auf die spezifischen Anforderungen der Anwender abgestimmt sind.

Die zentrale Forschungsfragen in dieser Arbeit lauten: 

\newcommand{\tab}{\hspace*{2em}}

\tab•	Welche Anforderungen haben Softwarearchitekten an Modellierungssprachen ?
\tab•	Welche Herausforderungen und Verbesserungspotenziale sehen Softwarearchitekten bei der Anwendung bestehender Modellierungssprachen?

Um die Forschungsfragen präzise beantworten zu können erfordert ein erster Schritt die Begrifflichkeiten zu klären.

Thematisch relevant sind die Begriffe:\\
 
\tab•	Softwarearchitektur \\
\tab•	Modellierungssprache \\
\tab•	Anforderungen\\


Das Ziel besteht darin, durch Interviews eine ein tiefes Verständnis der Nutzerperspektive zu gewinnen und Empfehlungen zu formulieren, die zur Optimierung bestehender Modellierungswerkzeuge beitragen können. Die Ergebnisse dieser Studie sollen dazu beitragen, die Entwicklung und Verbesserung von Modellierungswerkzeugen zu leiten, die ideal auf die spezifischen Anforderungen der Anwender abgestimmt sind.
Um dieses Ziel zu erreichen wird die im kommenden Abschnitt beschriebene Methodik angewandt.

\section{Aufbau der Arbeit und Methodik}

Hier soll der Arbeit und die Methodik erklärt werden








%%% Local Variables:
%%% mode: latex
%%% TeX-master: "main.tex"
%%% TeX-open-quote: "\\enquote{"
%%% TeX-close-quote: "}"
%%% LaTeX-csquotes-open-quote: "\\enquote{"
%%% LaTeX-csquotes-close-quote: "}"
%%% End: 

