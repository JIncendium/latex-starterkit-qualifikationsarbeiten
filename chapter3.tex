% !TEX root = main.tex
% !TEX TS-program = pdflatex
% !BIB program = biber


\lstset{language=C}

\chapter{Vorgehensmodell, Methodik der Arbeit}

Nachdem im Abschnitt 2 die theoretische Basis zu Softwarearchitekturmodellierung und Softwarearchitektur erläutert wurde, stellt sich nun die Frage, wie die Forschungsfragen im Rahmen dieser Masterarbeit am besten beantwortet werden können. Hierfür wurde der qualitative Forschungsansatz gewählt. Da die Analysegegenstände die Anforderungen und die Nutzung von Modellierungssprachen durch Softwarearchitekten sind, ist die geisteswissenschaftliche Methodik des Verstehens dem Analysegegenstand gegenüber angemessen und realitätsgerecht.\footnote{Vgl.  Siegfried Lamnek, Qualitative Sozialforschung, Beltz Verlagsgruppe, 2010, S. 27}

\textcolor{red}{ Noch Quelle Prüfen für nachfolgenden Abschnitt}
Nach Flick ist eine standardisierter Fragebogen sinnvoll bei festen Vorstellungen über den Untersuchungsgegenstand, wobei bei der qualitativen Forschung noch unbekanntes im Vordergrund steht. \footnote{Uwe Fick et al, Qualitative Forschung: Ein Handbuch,Rowolt Taschenbuch Verlag, 2017, S. 17  }
\textcolor{red}{  Quelle Prüfen für vorgeherhenden Abschnitt}
Für die maximale Ausschöpfung des spezifischen Informationspotentials wird eine qualitative Herangehensweise bevorzugt. Damit wird dem Prinzip der Offenheit Rechnung getragen. In der quantitativen Forschung wäre ein Kern der Forschung die Vergleichbarkeit. Diese wird in der qualitativen Forschung zu gunsten der Prinzipien Offenheit, Kommunikation, Gegenstandsangemessenheit und Reflexivität weniger in den Fokus gerückt.\footnote{Vgl. Jörg Strübing, Qualitative Sozialforschung: Eine Einführung, Oldenbourg Wissenschaftsverlag, 2013, S.20 ff}
Im folgenden werden die Gestaltung der durchgeführten qualitativen Forschung insbesondere die Erhebungs- und Auswertungsmethodik beschrieben. 

\section{Methode und Datenerhebung}

Das leitfadengestützte Experteninterview wurde als Erhebungsintrument der empirischen Untersuchung gewählt, da ausdrücklich das Wissen von ausgewählten Experten, hier Softwarearchitekten, für die Beantwortung der Forschungsfragen notwendig ist. Experten sind so zu verstehen, das diese eine Quelle von Spezialwissen des zu erforschenden Sachverhaltes darstellen. Sie stellen demnach nicht das Forschungsobjekt dar, sondern eröffnen durch ihr Wissen eine Perspektive auf das Forschungsobjekt. \footnote{Vgl. Jochen Flick, Laudel Gläser, Experteninterviews und qualitative Inhaltsanalyse, VS Verlag, 2010, S.12 }. Diese Methode wird innerhalb der qualitativen Verfahren der empirischen Sozialforschung als offene Befragung bezeichnet.\footnote{Vgl.N. Baur \& J.Blasius, Handbuch Methoden der empirischen Sozialforschung, Springer VS Wiesbaden, 2022, S.19} Um es durchzuführen ist es notwendig vor dem ersten Interview einen Leitfaden zu entwickeln, er beruht auf der bewussten methodologischen Entscheidung der maximalen Offenheit, da es keine Einschränkungen bei den Antworten des Interviewten gibt.\footnote{Vgl.Helfferrich in Baur \& J.Blasius, Handbuch Methoden der empirischen Sozialforschung, Springer VS Wiesbaden, 2022, S.877ff }

 

\section{qualitatives induktives empirisches Forschungsdesign}

Die Interviews sollen mit der qualitativen Inhaltsanalyse nach Mayring ausgewertet werden. Einen kurzen Überblick hierzu bildet die folgende Abbildung.

\begin{figure}[htb]
    \centering
    \includegraphics[width=0.7\textwidth]{Abbildung2inhaltsanalytischesAuflaubmodell}
    \caption{Allgemeines inhaltsanalystisches Ablaufmodell S.61 aus P. Mayrings Qualitative Inhaltsanalyse}
    \label{fig:Abbildung2inhaltsanalytischesAuflaubmodell}
\end{figure}

\section{Datenerhebung mithilfe qualitativer Interviews von Softwarearchitekten}

Die Fragestellungen die in den Experteninterviews besprochen werden, basieren auf den bereits im Kapitel 2 erörterten theoretischen Grundlagen und den Forschungsfragen. Diese Fragen befassen sich mit den Anforderungen an Modellierungssprachen und den Herausforderungen und Verbesserungsmöglichkeiten von Modellierungssprachen. Die Fragen beleuchten die praxiserfahrung der Experten mit dem Umgang von Modellierungssprachen. Die Interviews wurden mit Hilfe eines Leitfades geführt, dieser gliedert sich in fünf Abschnitte.
 
Der erste Abschnitt dient der Einführung des Interviewteilnehmers in das Thema des Interviews, hier wird die Vor dem Interview bereits eingeholte Zustimmung zur aufzeichnung erneut besätigt bevor der Ziel und das Zweck des Interview vorgestellt werden. Der Abschnitt Endet mit der Vorstellung des Experten zu seinem Tätigkeitsbereich und seiner Berufserfahrung.
Im zweiten Abschnitt geht es um die tägliche Arbeit des Experten im Rahmen seiner Architektentätigkeit, seine Arbeit mit Modellierungssprachen und der Übergangsfrage zum Thema "Wann beginnt für Sie Modellierung?"
Im dritten Abschnitt des Interviews werden die für den Architekten wichtigen Eigenschaften von Modellierungssprachen aus der Praxis beleuchtet. Der Leitfaden sieht hier mehrere Fragen zu verschiednen Anforderungen an Modellierungssprachen vor, ebenso zu den Kunden für die der Architekt seine Modellierungstätigkeit ausführt.
Die Fragen des vierten Abschnitts befassen sich mit der zweiten Forschungsfrage und stellen auf die Herausforderungen und Verbesserungspotentiale bei der Nutzung bestehender Modellierungsssprachen ab. Neben der Frage nach konkreten Problemen in Form von Bespielen aus der Praxis werden auch Fragen zur Optimierung verschiedener Aspekte von Modellierungssprachen gestellt.
Im letzten Abschnitt wird das Interview beendet mit Fragen nach Empfehlungen des Architekten oder allgemeinen Anmerkungswünschen des Architekten zum Thema. 

Im Rahmen der Masterarbeit wurden zehn Experteninterviews mit Softwarearchitekten geführt. Die Interviews fanden zwischen dem 13.11.2025 und dem 19.12.2025 statt und wurden alle mit der Videokonferenzsoftware Zoom getätigt. Um der Fragestellung gerecht zu werden, muss jeder Experte aktuelle Berufserfahrung im Softwarearchitektur Umfeld haben. Dieses Kriterium war Bedingung für die Einladung zum Interview. Die Auswahl erfolgte zunächst über berufliche Kontakte. Im IT-Arbeitsumfeld des Interviewers gibt es mehrere Software-Architekturboards mit Teams von Software-Architekten. Hier erfolgte die   


Die Video- und Audioaufzeichnugnen wurden OBS-Studio aufgenommen, mit dem VLC-Media-Player in Audioaufnahmen im mp3-Format umgewandelt und mit der 


\section{Begründung der Methodik}


\ 
