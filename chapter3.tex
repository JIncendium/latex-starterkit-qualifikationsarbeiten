% !TEX root = main.tex
% !TEX TS-program = pdflatex
% !BIB program = biber


\lstset{language=C}

\chapter{Vorgehensmodell, Methodik der Arbeit}

Nachdem in Abschnitt 2 die theoretische Basis zu Softwarearchitekturmodellierung und Softwarearchitektur erläutert wurde, widmet sich das nächste Kapitel der methodischen Herangehensweise zur Beantwortung der Forschungsfragen. Gegenstand wird daher der Grund der Methodikwahl das methodische Vorgehen sowie der Aufbau des Leitfadens sein. Schließlich wird die Art der Datenerhebung, die Durchführung der Interviews und Art der Datenauswertung dargestellt.





\section{Motivation  und Begründung der Methodik}

Grundsätzlich bestehen zwei unterschiedliche Herangehensweisen, um sich den Forschungsfragen zu nähern. Hier stehen sich die quantitativen und die qualitativen Methoden gegenüber. Beim quantitativen Forschungsansatz wird mit Methoden der Mathematik und Stochastik der Forschungsgegenstand erschlossen und versucht die gewonnen Ergebnisse in Zahlen auszudrücken. \footnote {Strübing, Jörg. Qualitative Sozialforschung: Eine komprimierte Einführung, Berlin, Boston: De Gruyter Oldenbourg, 2024. S. 4} 

Bei der quantitativen Forschung hingegen bestehen häufig bereits feste Vorstellungen über den Forschungsgegenstand.\footnote{Uwe Fick et al, Qualitative Forschung: Ein Handbuch,Rowolt Taschenbuch Verlag, 2017, S. 17 }
Ein Beispiel um quantitativ zu Forschen wäre die Nutzung eines vollstandardisierten Fragebogen. Hier werden die Daten aggregiert und entsprechend verarbeitet. \footnote {Strübing, Jörg. Qualitative Sozialforschung: Eine komprimierte Einführung, Berlin, Boston: De Gruyter Oldenbourg, 2024. S. 4} 
Demgegenüber sollte mit qualitativen Methoden gearbeitet werden, sobald Datenmaterial erschlossen werden soll, das sich mit formalen, quantifizierenden Auswertungen nicht angemessen erschließen lässt. Dies ist gerade bei umfangreichem Forschungsmaterial, wie es bei ausführlichen Beobachtungsprotokollen oder dem Ergebnis einer wörtlichen Abschrift eines Gesprächsverlaufes oftmals der Fall ist. \footnote {Strübing, Jörg. Qualitative Sozialforschung: Eine komprimierte Einführung, a.a.O. S. 4}

Für das hiesige Vorgehensmodell wurde der qualitative Forschungsansatz gewählt. Da Analysegegenstände die Anforderungen und die Nutzung von Modellierungssprachen durch Softwarearchitekten sind, ist die geisteswissenschaftliche Methodik des Verstehens dem Analysegegenstand gegenüber angemessen und realitätsgerecht.\footnote{Vgl. Siegfried Lamnek, Qualitative Sozialforschung, Beltz Verlagsgruppe, 2010, S. 27} 

Für die maximale Ausschöpfung des spezifischen Informationspotentials wird eine qualitative Herangehensweise bevorzugt, um dem Prinzip der Offenheit Rechnung zu tragen. Anders als in der quantitativen Forschung, deren Kern die Vergleichbarkeit darstellt, werden in der qualitativen Forschung die Prinzipien Offenheit, Kommunikation, Gegenstandsangemessenheit und Reflexivität in den Fokus gerückt.\footnote{Vgl. Jörg Strübing, Qualitative Sozialforschung: Eine Einführung, Oldenbourg Wissenschaftsverlag, 2013, S.20 ff} 

Im Folgenden wird die Gestaltung der durchgeführten qualitativen Forschung, insbesondere die Erhebungs- und Auswertungsmethodik beschrieben. 

\section{ Qualitatives induktives empirisches Forschungsdesign}
Die Auswertung von qualitativen Interviews wird mit der Methodik, die Mayring entworfen und als qualitative Inhaltsanalyse bezeichnete, durchgeführt. Mayring hat für die qualitative Inhaltsanalyse ein allgemeines Ablaufmodell erstellt. Diese Systematik, wie in Abbildung 3.1 dargestellt, wird im Folgenden näher erörtert, da Sie die Grundlage für die Auswertung der Experteninterviews darstellt.

\begin{figure}[htbp]
    \centering
    \includegraphics[width=0.7\textwidth]{Abbildung2inhaltsanalytischesAuflaubmodell}
    \caption{Allgemeines inhaltsanalystisches Ablaufmodell S.61 aus P. Mayrings Qualitative Inhaltsanalyse}
    \label{fig:Abbildung2inhaltsanalytischesAuflaubmodell}
\end{figure}

Die Systematik beginnt in einem ersten Schritt mit der Festlegung des Materials. In diesem Schritt wird die Auswahl fixiert, „welches Material der Analyse zugrunde gelegt werden soll“\footnote {P. Mayring Qualtitative Inhaltsanalyse Grundlagen und Techniken, Beltz, 2015, S54}. 

Im zweiten Schritt, der Analyse der Entstehungssituation, wird beschrieben wie, von wem und unter welchen Bedingungen das Material produziert wird und zu welchem Zweck. “\footnote {P. Mayring Qualtitative Inhaltsanalyse Grundlagen und Techniken, Beltz, 2015, S55}. Im dritten Schritt werden die formalen Charakteristika des Materials beschrieben. Dies stellt in der Regel einen niedergeschriebenen Text dar. “\footnote {P. Mayring Qualtitative Inhaltsanalyse Grundlagen und Techniken, Beltz, 2015, S55}.

Der vierte Schritt Richtung der Analyse dient der Klärung, unter welcher Perspektive das Datenmaterial analysiert werden soll. Von den von Mayring aufgezeigten Möglichkeiten wird sich an dieser Stelle für die Beschreibung des im Material beschriebenen Gegenstandes entschieden. Die weiteren Möglichkeiten sind die Analysepunkte, die sich mit dem psychischen Zustand des Kommunikationspartners oder der Vorgehensweise in der Literaturwissenschaft befassen. “\footnote {P. Mayring Qualtitative Inhaltsanalyse Grundlagen und Techniken, Beltz, 2015, S58} Da es sich in dieser Arbeit weder um eine Literarturwissenschaftliche Forschungsarbeit noch um eine Analyse der Psyche der Interviewteilnehmer handelt, sind diese Methoden nicht geeignet um die Forschungsfragen zu beantworten. 

In einem fünften Schritt, die Mayring theoretische Differenzierung der Fragestellung nennt, wird das Textmaterial mit der bestehenden Theorie der Thematik in Verbindung gebracht. Auf diese Weise soll eine Verknüpfung der bisherigen Forschung zum Forschungsgegenstand erfolgen. “\footnote {P. Mayring Qualtitative Inhaltsanalyse Grundlagen und Techniken, Beltz, 2015, S59}. 
Im sechsten Schritt wird die passende Analysetechnik bestimmt und das konkrete Ablaufmodell festgelegt. Hier bestehen mehrere Grundformen, die miteinander kombinierbar sind:“\footnote { vgl. P. Mayring Qualtitative Inhaltsanalyse Grundlagen und Techniken, Beltz, 2015, S67}. Zusammenfassung, Explikation und Strukturierung. 

Mithilfe der Zusammenfassung wird das Textmaterial durch Kürzen, Reduzieren und Paraphrasieren so reduziert, dass die wesentlichen Inhalte bestehen bleiben und das Datenmaterial an Übersichtlichkeit gewinnt. Die Explikation wird genutzt, wenn Textpassagen ergänzt werden müssen, um das Verständnis derselbigen zu verdeutlichen. Mit der Strukturierung als dritte Analysetechnik, besteht die Möglichkeit der Kategorienbildung und der Einordnung des vorliegenden Materials. “\footnote { vgl. P. Mayring Qualtitative Inhaltsanalyse Grundlagen und Techniken, Beltz, 2015, S67}.

Im siebten Schritt des Vorgehensmodells werden die Analyseeinheiten definiert, die Größe der zu analysierenden Textpassagen und Auswertungsreihenfolge werden festgelegt. 

Im achten Schritt werden die Analyseschritte gemäß des Ablaufmodells und die Rücküberprüfung des Kategoriensystems vorgenommen. Die Rücküberprüfung stellt sicher, dass die zusammengefassten gewonnen Aussagen noch das Ausgangsmaterial repräsentieren. 

Im neunten Schritt werden nun die Ergebnisse zusammengestellt und unter Bezugnahme auf die Forschungsfragen interpretiert.\footnote {. vgl. P. Mayring Qualtitative Inhaltsanalyse Grundlagen und Techniken, Beltz, 2015, S61}
Im abschließenden zehnten Schritt erfolgt mit der Anwendung der Gütekriterien Reliabiltät und Validität eine Einschätzung der Analyse auf Tauglichkeit \footnote {Vgl. P. Mayring Qualtitative Inhaltsanalyse Grundlagen und Techniken, Beltz, 2015, S.123f}

\section{Zusammenfassung nach Mayring}

Für die Auswertung der Experteninterviews gewählt wurde die Analysetechnik „Zusammenfassung“ . Um das Ziel der Zusammenfassung zu erreichen hat Mayring ebenfalls eine Vorgehensweise festgelegt, die für diese Inhaltsanalyse umgesetzt wird. Es handelt sich um folgendes siebenstufige Vorgehen:

\begin{enumerate}
    \item Stufe: Die Analyseeinheiten werden bestimmt.
    \item Stufe: Die inhaltstragenden Textstellen werden regelgeleitet paraphrasiert.
    \item Stufe: Das angestrebte Abstraktionsniveau wird bestimmt. Die Paraphrasen werden unter diesem Abstraktionsniveau generalisiert.
    \item Stufe: Erste Reduktion durch Auswahl und Streichung von bedeutungsgleichen Paraphrasen.
    \item Stufe: Zweite Reduktion durch Bündelung, Konstruktion und Integration auf dem angestrebten Abstraktionsniveau.
    \item Stufe: Zusammenstellung der Aussagen als Kategoriesystem.
    \item Stufe: Rücküberprüfung des zusammenfassenden Kategoriesystems am Ausgangsmaterial.
\end{enumerate}

Bevor im Ergebnisteil die Haupt- und Unterkategorien vorgestellt werden wird in den nächsten Abschnitten das Methodische Vorgehen und der Leitfaden vorgestellt.

\section{Methodisches Vorgehen }

Zur Beantwortung der Forschungsfragen ist in der vorliegenden Arbeit ausdrückliches Wissen von ausgewählten Experten, hier Softwarearchitekten, erforderlich. Aus diesem Grund wurde das leitfadengestützte Experteninterview als Erhebungsinstrument der empirischen Untersuchung gewählt. Experten sind so zu verstehen, dass diese eine Quelle von Spezialwissen des zu erforschenden Sachverhaltes darstellen. Sie stellen demnach nicht selbst das Forschungsobjekt dar, sondern eröffnen durch ihr Wissen eine Perspektive auf das Forschungsobjekt. \footnote{Vgl. J. Flick, L. Gläser, Experteninterviews und qualitative Inhaltsanalyse, VS Verlag, 2010, S.12 } Um diese Perspektive zu erschließen wird die Methode genutzt, die innerhalb der qualitativen Verfahren der empirischen Sozialforschung als offene Befragung bezeichnet wird.\footnote{Vgl.N. Baur \& J.Blasius, Handbuch Methoden der empirischen Sozialforschung, Springer VS Wiesbaden, 2022, S.19} Um das Verfahren durchzuführen ist es notwendig vor dem ersten Interview einen Leitfaden zu entwickeln. Dieser beruht auf der bewussten methodologischen Entscheidung der maximalen Offenheit, da es keine Einschränkungen bei den Antworten des jeweiligen Interviewpartners gibt.\footnote{Vgl.Helfferrich in Baur \& J.Blasius, Handbuch Methoden der empirischen Sozialforschung, Springer VS Wiesbaden, 2022, S.877ff }

\section{Aufbau des Leitfadens}
Der folgende Abschnitt beschreibt den Aufbau des Leitfadens, der die Grundlage des leitfadengestützten Experteninterviews darstellt und die offene Befragung des Experten ermöglicht. sowie den Entscheidungsprozess für die Auswahl der Fragen.
Der Einsatz eines Leitfadens ermöglicht es, die offenen Fragen in den Gesprächen mit den Softwarearchitekten zu strukturieren. Gleichzeitig kann auch eine gewisse Vergleichbarkeit der Interviews geschaffen werden.\footnote{C. Helfferich, in Handbuch der empirischen Sozialforschung, Springer VS, 2022, S. 875-892}

Die Interviewfragen des Leitfadens wiederum werden in Orientierung an den Phasen eines wissenschaftlichen Interviews (Einleitung, Hauptteil und Abschluss) sowie gleichzeitig nach inhaltlichen Bereichen sortiert. 
Um die Fragestellungen und den Zeitrahmen des Leitfadens zu prüfen ist ein Pretest durchgeführt worden. Der Pretest ist unter identischen Bedingungen wie die Hauptinterviews geführt worden. Der Pretest soll darüber hinaus die Sinnhaftigkeit der Reihenfolge der Fragen, der Verständlichkeit der Fragen sowie potentielle Fehlerquellen in der Interviewdurchführung offenlegen. Nach der Durchführung des Pretests fand eine kleine Anpassung im Schlussteil des Leitfadens statt. 
Der vollständige Leitfaden findet sich im Anhang. Die 19 Fragen des Leitfadens werden auf die folgenden Abschnitte aufgeteilt:

\begin{enumerate}
    \item Einführung
    \item Allgemeiner Hintergrund
    \item Anforderungen an Modellierungssprachen
    \item Herausforderungen und Verbesserungspotenziale
    \item Abschlussfragen
\end{enumerate}

Der erste Abschnitt dient der Einführung des Befragten in das Thema des Interviews. Hier wird die vor dem Interview bereits eingeholte Zustimmung zur Aufzeichnung erneut bestätigt, bevor das Ziel und der Zweck des Interviews vorgestellt werden. Der Abschnitt endet mit der Vorstellung des Experten zu seinem Tätigkeitsbereich und seiner Berufserfahrung.

Der zweite Teil des Leitfadens fragt allgemein nach der Arbeit des Softwarearchitekten, mit welchen Werkzeugen (Modellierungssprachen) gearbeitet wird und ab wann Modellierung notwendig wird. Die im hier enthaltenen Fragen sollen erzählgenerierende Wirkung entfalten. Die Aufforderung über die Tätigkeit zu sprechen folgt der Empfehlung von Helfferich.\footnote{C. Helfferich, in Handbuch der empirischen Sozialforschung, Springer VS, 2022, S. 884} Damit sind auch Fragen nach der Dauer der Tätigkeit und der fachlichen Ausrichtung eingeschlossen.

Den Kern des Leitfadens bilden der dritte und vierte Abschnitt. Für die zentralen Forschungsfragen ist jeweils ein Abschnitt im Interview Vorgesehen
Die Fragen des dritten Abschnittes fokussieren sich in ihren jeweiligen Formulierungen schwerpunktmäßig auf die Anforderungen an Modellierungssprachen. Zunächst soll in Erfahrung gebracht werden, welche Anforderungen im Praxisalltag besonders relevant sind. Mit den Fragen nach den verschiedenen Anforderungstypen und der Einordnung der Priorität wird auf die in Kapitel 2 definierte Anforderungssystematik (2.5) zurückgegriffen. Es wird explizit nach der Bedeutung von formalen Anforderungen, anwenderbezogenen Anforderungen und anwendungsbezogenen Anforderungen für die Softwarearchitekten gefragt.

Der vierte Abschnitt des Leitfadens widmet sich der zweiten Forschungsfrage. Hier wird durch die Frage nach den Herausforderungen bei der Nutzung bestehender Modellierungssprachen und den Missverständnissen noch einmal der konkrete Blick des Architekten auf die Modellierungssprachen erfragt. Die Schärfung des Kontextes erfolgt über die Nachfragen nach spezifischen Verbesserungspotentialen in Bezug auf Semantik, Notation und Toolunterstützung.
Mit der Frage nach speziellen Anwendungsfällen, in denen die bestehenden Sprachen Probleme haben wird, wird versucht, den Blick auf die Anforderungen zu verbessern. Zuletzt folgt die offene Frage nach Verbesserungspotentialen.

Der Schlussabschnitt des Leitfadens gibt dem Befragten die Möglichkeit noch einen freien Beitrag zum Themenkomplex zu geben. Mit der Frage nach „Best practices“ und persönlichen Empfehlungen für die Auseinandersetzung mit Softwarearchitektur wird der Inhaltliche Teil des Interviews abgeschlossen.

Der Leitfaden dient darüber hinaus dem Forschenden zur thematischen Fokussierung und der Organisation des eigenen Wissens.

\section{Datenerhebung und Stichprobe}

Die Abschnitte gehen auf die Stichprobe und die Entwicklung des Interviewleitfadens ein.

\subsection{Stichprobe}

Die Expertengewinnung und -auswahl ist für die Befragung eine zentrale praktische Hürde, darüber hinaus hat dieser Arbeitsschritt eine maßgebliche Bedeutung auf die Güte der Datenerhebung. \footnote{Vgl. N. Baur \& J.Blasius, Handbuch Methoden der empirischen Sozialforschung, Springer VS Wiesbaden, 2022, S.1 Handbuch methoden}

Nach Akremi(2022) ist die Auswahl der Stichprobe entscheidend für die Güte der Forschungsergebnisse. Die folgende Tabelle \ref{tab:Stichprobenwahl}, in Anlehnung an Akremi, zeigt die Überlegungen für die Auswahl der Interviewpartner.

\begin{table}[htbp]
    \centering
    \includegraphics[width=0.9\textwidth]{Kapitel_3_Tab_Stichprobenwahl_nach Akremi.png}
    \caption{Stichprobenwahl(Eigene Darstellung nach Akremi, 2022)}
    \label{tab:Stichprobenwahl}
\end{table}

Eine genaue Größe der Stichprobe ist für qualitative Forschung nicht festgelegt,
Helfferich empfiehlt eine Stichprobengröße von mindestens sechs Interviews bei hermeneutischen Interpretationen.\footnote{C. Helfferich, Die Qualität qualitativer Daten Manual für die Durchführung qualitiativer Interviews,Springer Fachmedien Wiesbaden GmbH, 2011, S.175}

In der hiesigen Arbeit wurde für die Experteninterviews ein Stichprobenumfang von zehn Experten gewählt. Der befragte Softwarearchitekt des Pretests und die neun weiteren Softwarearchitekten. Da der Pretest nur marginale Änderungen am Leitfaden nötig machte, konnte das Interview des Pretests ohne Einschränkung mit in die Auswertung einbezogen werden. 

Das ausschlaggebende Auswahlkriterium war die aktuelle Tätigkeit als Softwarearchitekt. Dies wurde durch die Rolle und die Position im jeweiligen Unternehmen gewährleistet. Ebenfalls in die Überlegung mit einbezogen wurde, ob es sinnvoll sein kann die Experten in unterschiedlichen Hierarchiestufen und nicht nur auf der Arbeitsebene zu suchen. Hier überzeugte der Gedanke, dass der Perspektivwechsel beim Hierarchiewechsel vermutlich auch die Nutzung von Modellierungssprachen beeinflusst. 

Dies wiederum kann andere Anforderungen an Modellierungssprachen zum Vorschein bringen. Selbst bei exakt gleich gearteter Nutzung der Modellierungssprachen hat eine Softwarearchitekt mit Führungsverantwortung alleine hierarchisch bedingt einen anderen Adressatenkreis, dem die Arbeitsergebnisse präsentiert werden müssen, als ein Softwarearchitekt auf der operativen Arbeitsebene.

Die nachstehende Tabelle \ref{tab:Untersuchungspopulationarchitektur} gibt einen Überblick über die Interviewpartner.

\begin{table}[htbp]
    \centering
    \includegraphics[width=0.9\textwidth]{Kapitel_3_Tab_Untersuchungspopulation ExpertenInterview.png}
    \caption{Untersuchungspopulation Experteninterview(Eigene Darstellung)}
    \label{tab:Untersuchungspopulationarchitektur}
\end{table}

\subsection{Vorbereitung und Durchführung der Interviews}

Die vorbereitenden Tätigkeiten erfolgten in parallelen Arbeitsschritten. Hierzu gehörten die Erstellung, Anpassung und Freigabe des Leitfadens. Neben dieser Tätigkeit erfolgte die erste Kontaktaufnahme bei potentiellen Interviewpartnern per persönlicher Vorstellung oder Skype-Anruf. In einem Fall wurde der Erstkontakt per Emailanfrage hergestellt. Bei der Vorstellung wurden die potentiellen Interviewpartner über die Masterarbeit, das Ziel und den Nutzen des Interviews ebenso wie die geplante Dauer informiert. Den Experten wurde in diesem Zusammenhang die Anonymisierung der Expertennamen in der Transkription zugesichert.

Mit dem Hinweis auf die Anonymisierung sollte im Interview eine offene und vertrauliche Gesprächsumgebung geschaffen werden. Ebenso ist in der Darstellung der Auswertung eine Zuordnung einzelner Aussagen durch Dritte nicht mehr möglich. Damit waren alle Interviewpartner einverstanden. 
Mit der Übergabe der Audioaufnahmen an den Lehrstuhl im Rahmen der Qualifikationsarbeitsabgabe waren neun Experten einverstanden, nachdem erklärt wurde, dass eine Veröffentlichung der Masterarbeit nicht vorgesehen ist. 
Da sich in den Vorgesprächen mit dem zehnten Experten herausgestellt hat, dass dieser die Nutzung von Modellierungssprachen sehr kritisch sieht, war das Interesse an dieser Perspektive mit Hinblick auf die Forschungsfrage sehr ausgeprägt. Nach Rücksprache mit dem Masterarbeitsbetreuer des Lehrstuhls konnte das Experteninterview mit dem Experten, der der Übergabe der Audioaufnahme nicht zugestimmt hat, dennoch geführt werden. Die Audioaufnahme wurde, entsprechend der Vereinbarung mit diesem Experten, nicht dem Lehrstuhl übergeben. 

Mit der Zusage zum Interview wurden die Einladungen verschickt. In diesen wurde erneut auf die Vertraulichkeit hingewiesen. Auch wurde über das Thema des Interviews kurz informiert. Den Interviewteilnehmern wurde im Vorfeld nur auf expliziten Wunsch der Leitfaden vorab zugeschickt. Damit sollte ein möglichst natürlicher Fragenverlauf und unvorbereitete Antworten generiert werden. Lediglich ein Interviewteilnehmer wollte vorab den Leitfaden für das Interview erhalten. Dieser wurde mit einer Vorlaufzeit von einer Woche dem Interviewpartner zugeschickt. 

Für die Interviews wurde das Videokonferenztool Zoom genutzt. in einem Fall musste aufgrund von anbieterseitigen Serverproblemen als Alternative auf Skype for Business zurückgegriffen werden. Für die zur Aufnahme der Interviews wurde die Software OBS-Studio verwendet. 
Die Interviews fanden zwischen dem 13.11.2025 und dem 19.12.2025 statt. Die Gesamtdauer der zehn Interviews betrug 8 Stunden 15 Minuten. 

\subsection{Transkription}

Da es für das Transkribieren keine allgemein festgehaltenen Regeln gibt, ist die Aufstellung und Einhaltung eigener Transkriptionsregeln ein Teil der wissenschaftlichen Vorgehensweise.\footnote{J. Gläser, G. Laudel, Experteninterviews und Qualitative Inhaltsanalyse, VS Verlag 2010 S.193}
Transkriptionsregeln wie sie z. B. Von Kallmeyer und Schütze empfohlen werden, werden nicht angewandt. Hier besteht ein großer Teil der Transkriptionsregularik darin, die gesprochene Stimmlage, Pausen und Intonation des Befragten abzubilden. \footnote{Vgl. S. Fuß, U. Karbach, Grundlagen der Transkription: Eine praktische Einführung, Verlag Barbara Budrich GmbH, 2019, S.35f} Für die Auswertung der hier gestellten Forschungsfragen ist der Inhalt der Antworten von größter Bedeutung. Die ausdifferenzierte Kennzeichnung von der Art und Weise wie der Befragte Antwortet rückt in den Hintergrund.

Daher orientieren sich die hier angewandten Transkriptionsregeln, an den von Kuckartz aufgestellten Regeln. \footnote{Vgl. S. Fuß, U. Karbach, Grundlagen der Transkription: Eine praktische Einführung, Verlag Barbara Budrich GmbH, 2019, S.30 } 

\begin{enumerate}
    \item Es wird wörtlich transkribiert.
    \item Sprache und Interpunktion werden leicht geglättet, jedoch wird das Sprechen „ohne Punkt und Komma“ nicht in kurze Sätze unterteilt.
    \item Deutlich längere Pausen werden durch Klammern (\dots) gekennzeichnet.
    \item Alle Angaben, die einen Rückschluss auf eine befragte Person erlauben, werden anonymisiert.
    \item Absätze der interviewenden Person und des befragten Experten werden klar voneinander abgehoben.
    \item Jeder Sprechbeitrag wird als eigener Absatz transkribiert, um die Lesbarkeit des Transkripts zu erhöhen.
    \item Relevante nonverbale Aktivitäten werden im Transkript festgehalten, z.\,B. \dots
    \item Unverständliche Wörter werden durch (unverständlich) als solche kenntlich gemacht.
\end{enumerate}

Für die Umwandlung und Komprimierung der Audiodateien wurde der VLC-Media-Player verwendet. Die Transkription wurde von der Software Whisper-AI unterstützt. Hierzu wurden die mp3-Audiodatein von Whisper-AI transkribiert. Im Anschluss wurde das Transkript mit den Audiodateien verglichen und fehlerhafte Teile ausgetauscht. Ein häufig vorkommendes Beispiel für eine inkorrekte Transkription ist die von der Software erkannte Textpassage als „DAV-Verbund“. Dies wurde dann durch „DRV-Bund“ ersetzt, da die Interviewteilnehmer in diesem Kontext von ihrem Arbeitgeber, der DRV-Bund, gesprochen haben. 

Im folgenden Kapitel werden die Ergebnisse präsentiert, die auf der Auswertung der Interviews gemäß der zuvor beschriebenen methodischen Vorgehensweise basieren.