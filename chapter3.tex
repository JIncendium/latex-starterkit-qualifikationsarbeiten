% !TEX root = main.tex
% !TEX TS-program = pdflatex
% !BIB program = biber


\lstset{language=C}

\chapter{Vorgehensmodell, Methodik der Arbeit}

Nachdem im Abschnitt 2 die theoretische Basis zu Softwarearchitektur Modellierung und Softwarearchitektur erläutert wurde, stellt sich nun die Frage, wie die Forschungsfragen im Rahmen dieser Masterarbeit am besten beantwortet werden können. Hierfür wurde der qualitative Forschungsansatz gewählt. Da der Analysegegenstand Softwarearchitekten und ihre Nutzung von Modellierungssprachen sind ist die geistenswissenschaftliche Methodik des Verstehens dem Analysegegenstand gegenüber angemessen und realitätsgerechter.\footnote{Vgl.  Siegfried Lamnek, Qualitative Sozialforschung, Beltz Verlagsgruppe, 2010, S. 27}

\textcolor{red}{ Noch Quelle Prüfen für nachfolgenden Abschnitt}
Nach Flick ist eine standardisierter Fragebogen sinnvoll bei festen Vorstellungen über den Untersuchenunsgegenstand, wobei bei der qualitativen Forschung noch unbekanntes im Vordergrund steht. \footnote{Uwe Fick et al, Qualitative Forshcung: Ein Handbuch,Rowolt Taschenbuch Verlag, 2017, S. 17  }
\textcolor{red}{  Quelle Prüfen für vorgeherhenden Abschnitt}
Für die maximale Ausschöpfung des spezifischen Informationspotentials wird eine qualitative Herangehensweise bevorzugt. Damit wird dem Prinzip der Offenheit Rechnung getragen. In der quantitativen Forschung wäre ein Kern der Forschung die Vergleichbarkeit. Diese wird in der qualitativen Forschung zu gunsten der Prinzipien Offenheit, Kommunikation, Gegesstandsangemessenheit und Reflexivität weniger in den Fokus gerückt.\footnote{Vgl. Jörg Strübing, Qualitative Sozialforschung: Eine Einführung, Oldenbourg Wissenschaftsverlag, 2013, S.20 ff}

\section{qualitatives induktives empirisches Forschungsdesign}

Die Interviews sollen mit der qualitativen Inhaltsanalyse nach Mayring ausgewertet werden. Einen kurzen Überblick hierzu bildet die folgende Abbildung.

\begin{figure}[htb]
    \centering
    \includegraphics[width=0.7\textwidth]{Abbildung2inhaltsanalytischesAuflaubmodell}
    \caption{Allgemeines inhaltsanalystisches Ablaufmodell S.61 aus P. Mayrings Qualitative Inhaltsanalyse}
    \label{fig:Abbildung2inhaltsanalytischesAuflaubmodell}
\end{figure}

\section{Datenerhebung mithilfe qualitativer Interviews von Softwarearchitekten}

\section{Begründung der Methodik}


\ 
