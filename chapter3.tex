% !TEX root = main.tex
% !TEX TS-program = pdflatex
% !BIB program = biber


\lstset{language=C}

\chapter{Vorgehensmodell, Methodik der Arbeit}

Nachdem im Abschnitt 2 die theoretische Basis zu Softwarearchitekturmodellierung und Softwarearchitektur erläutert wurde, stellt sich nun die Frage, wie die Forschungsfragen im Rahmen dieser Masterarbeit am besten beantwortet werden können. Im Folgenden wird auf das methodische Vorgehen, den Grund der Methodikwahl sowie die Entwicklung des Leitfadens eingegangen. Gefolgt von der Art der Datenerhebung und der Datenauswertung.
Für das Vorgehensmodell wurde der qualitative Forschungsansatz gewählt. Da die Analysegegenstände die Anforderungen und die Nutzung von Modellierungssprachen durch Softwarearchitekten sind, ist die geisteswissenschaftliche Methodik des Verstehens dem Analysegegenstand gegenüber angemessen und realitätsgerecht.\footnote{Vgl. Siegfried Lamnek, Qualitative Sozialforschung, Beltz Verlagsgruppe, 2010, S. 27}
Nach Flick ist ein standardisierter Fragebogen sinnvoll bei festen Vorstellungen über den Untersuchungsgegenstand, wobei bei der qualitativen Forschung noch unbekanntes im Vordergrund steht. \footnote{Uwe Fick et al, Qualitative Forschung: Ein Handbuch,Rowolt Taschenbuch Verlag, 2017, S. 17 }
\textcolor{red}{ Inhaltlich überarbeiten in sinnvollere Reihenfolge bringen}
\textcolor{red}{ Quelle Prüfen für vorgeherhenden Abschnitt}
Für die maximale Ausschöpfung des spezifischen Informationspotentials wird eine qualitative Herangehensweise bevorzugt. Damit wird dem Prinzip der Offenheit Rechnung getragen. In der quantitativen Forschung wäre ein Kern der Forschung die Vergleichbarkeit. Diese wird in der qualitativen Forschung zu gunsten der Prinzipien Offenheit, Kommunikation, Gegenstandsangemessenheit und Reflexivität weniger in den Fokus gerückt.\footnote{Vgl. Jörg Strübing, Qualitative Sozialforschung: Eine Einführung, Oldenbourg Wissenschaftsverlag, 2013, S.20 ff}
Im folgenden werden die Gestaltung der durchgeführten qualitativen Forschung insbesondere die Erhebungs- und Auswertungsmethodik beschrieben. 






\section{Begründung der Methodik}


Das leitfadengestützte Experteninterview wurde als Erhebungsinstrument der empirischen Untersuchung gewählt, da ausdrücklich das Wissen von ausgewählten Experten, hier Softwarearchitekten, für die Beantwortung der Forschungsfragen notwendig ist. Experten sind so zu verstehen, dass diese eine Quelle von Spezialwissen des zu erforschenden Sachverhaltes darstellen. Sie stellen demnach nicht das Forschungsobjekt dar, sondern eröffnen durch ihr Wissen eine Perspektive auf das Forschungsobjekt. \footnote{Vgl. J. Flick, L. Gläser, Experteninterviews und qualitative Inhaltsanalyse, VS Verlag, 2010, S.12 }. Diese Methode wird innerhalb der qualitativen Verfahren der empirischen Sozialforschung als offene Befragung bezeichnet.\footnote{Vgl.N. Baur \& J.Blasius, Handbuch Methoden der empirischen Sozialforschung, Springer VS Wiesbaden, 2022, S.19} Um es durchzuführen ist es notwendig vor dem ersten Interview einen Leitfaden zu entwickeln, er beruht auf der bewussten methodologischen Entscheidung der maximalen Offenheit, da es keine Einschränkungen bei den Antworten des Interviewten gibt.\footnote{Vgl.Helfferrich in Baur \& J.Blasius, Handbuch Methoden der empirischen Sozialforschung, Springer VS Wiesbaden, 2022, S.877ff }



\section{Entwicklung des Leitfadens}

Wie im vorherigen Abschnitt festgehalten, ist das leitfadengestützte Experteninterview das geeignetste Mittel um die Forschungsfrage näher zu beleuchten. Der folgende Abschnitt beschreibt die Entwicklung des Leitfadens, sowie den Entscheidungsprozess für die Auswahl der Fragen.
Um die offenen Fragen in den Gesprächen mit den Softwarearchitekten zu strukturieren, ergibt sich die Notwendigkeit eines Leitfadeneinsatzes. Hierdurch soll gleichzeitig auch eine gewisse Vergleichbarkeit der Interviews geschaffen werden.\footnote{C. Helfferich, in Handbuch der empirischen Sozialforschung, Springer VS, 2022, S. 875-892}

Die Interviewfragen werden in Orientierung an den Phasen eines wissenschaftlichen Interviews (Einleitung, Hauptteil und Abschluss) sowie gleichzeitig nach inhaltlichen Bereichen sortiert. 
Um die Fragestellungen und den Zeitrahmen zu prüfen ist ein Pre-test durchgeführt worden. Der Pretest soll darüber hinaus die Sinnhaftigkeit der Fragen-Reihenfolge, Verständlichkeit der Fragen und potentielle Fehlerquellen in der Interviewdurchführung offenlegen. 

Der erste Abschnitt dient der Einführung des Befragten in das Thema des Interviews, hier wird die, vor dem Interview bereits eingeholte Zustimmung zur Aufzeichnung erneut bestätigt, bevor das Ziel und der Zweck des Interviews vorgestellt werden. Der Abschnitt Endet mit der Vorstellung des Experten zu seinem Tätigkeitsbereich und seiner Berufserfahrung.

Der zweite Teil des Leitfadens fragt allgemein nach der Arbeit des Softwarearchitekten, mit welchen Werkzeugen (Modellierungssprachen) gearbeitet wird und ab wann Modellierung notwendig wird. Die im Einleitungsteil enthaltenen Fragen sollen erzählgenerierende Wirkung entfalten. Die Aufforderung über die Tätigkeit zu sprechen folgt der Empfehlung von Helfferich.\footnote{C. Helfferich, in Handbuch der empirischen Sozialforschung, Springer VS, 2022, S. 884} Damit sind auch Fragen nach der Dauer der Tätigkeit und der fachlichen Ausrichtung eingeschlossen.
Der Hauptteil ist in wurde in zwei Teile getrennt, somit ist für die zentralen Forschungsfragen jeweil ein Teil im Interview vorgesehen.

Die Fragen des ersten Abschnittes des Hauptteils fokussieren sich ihren jeweiligen Formulierungen Schwerpunktmäßig auf die Anforderungen an Modellierungssprachen. Zunächst soll in Erfahrung gebracht werden welche Anforderungen im Praxisalltag besonders relevant sind. Mit den Fragen nach den verschiedenen Anforderungstypen und der Einordnung der Priorität, wird auf die in Kapitel 2 definierte Anforderungssystematik (2.5) zurückgegriffen. Es wird explizit nach der Bedeutung von formalen Anforderungen, anwenderbezogenen Anforderungen und anwendungsbezogenen Anforderung für die Softwarearchitekten gefragt.

Der zweite Abschnitt des Hauptteils widmet sich der zweiten Forschungsfrage. Hier wird durch die Frage nach den Herausforderungen bei der Nutzung bestehender Modellierungssprachen und den Missverständnissen noch einmal der konkrete Blick des Architekten auf die Modellierungssprachen erfragt. Die Schärfung des Kontextes erfolgt über die Nachfragen nach spezifischen Verbesserungspotentialen in Bezug auf Semantik, Notation und Toolunterstützung. Mit der Frage nach speziellen Anwendungsfällen, in denen die bestehenden Sprachen Probleme haben wird, versucht den Blick auf die Anforderungen zu verbessern. Bevor zuletzt die offene Frage nach Verbesserungspotentialen gestellt wird.

Der Schlussteil des Leitfadens wird von der Frage nach der perfekten Modellierungssprache geprägt. Dies gibt dem Softwarearchitekten die Möglichkeit, bereits Gesagtes zusammenzufassen, zu ergänzen oder zu ersetzen und bildet somit eine offene Reaktionsmöglichkeit des Interviewten. 
Der Leitfaden dient darüber hinaus dem Forschenden zur thematischen Fokussierung und der Organisation des eigenen Wissens. Der Leitfaden wird vollständig im Anhang abgebildet.



\section{Datenerhebung und Stichprobe}



\subsection{Stichprobe}

Die Expertengewinnung und -auswahl ist für die Befragung eine zentrale praktische Hürde, darüber hinaus hat dieser Arbeitsschritt eine maßgebliche Bedeutung auf die Güte der Datenerhebung. \footnote{Vgl. N. Baur \& J.Blasius, Handbuch Methoden der empirischen Sozialforschung, Springer VS Wiesbaden, 2022, S.1 Handbuch methoden}
Nach Akremi(2022) ist die Auswahl der Stichprobe entscheidend für die Güte der Forschungsergebnisse. Die folgende Tabelle 3.1, in Anlehnung an Akremi, zeigt die Überlegungen für die Auswahl der Interviewpartner.


\textcolor{red}{hier Tabelle einfügen Stichprobenwahl eigene Darstellung (als jpg Hinterlegen}

Eine genaue Größe der Stichprobe ist für qualitative Forschung nicht festgelegt,
Helfferich empfiehlt eine Stichprobengröße von mindestens sechs Interviews bei hermeneutischen Interpretationen.\footnote{C. Helfferich, Die Qualität qualitativer Daten Manual für die Durchführung qualitiativer Interviews,Springer Fachmedien Wiesbaden GmbH, 2011, S.175}
Es wurde für die Experteninterviews ein Stichprobenumfang von 10 Experten gewählt. Das ausschlaggebende Auswahlkriterium war die aktuelle Tätigkeit als Softwarearchitekt. Dies wurde durch die Rolle und die Position im jeweiligen Unternehmen gewährleistet.

Ebenfalls in die Überlegung mit einbezogen wurde, ob es sinnvoll sein kann die Experten in unterschiedlichen Hierarchiestufen und nicht nur auf der Arbeitsebene zu suchen. Hier überzeugte der Gedanke, dass der Perspektivwechsel beim Hierarchiewechsel, vermutlich auch die Nutzung von Modellierungssprachen beeinflusst. Dies Wiederrum kann andere Anforderungen an Modellierungssprachen zum Vorschein bringen. Selbst bei exakter gleichgearteter Nutzung der Modellierungssprachen, hat eine Führungsrolle einen anderen Adressatenkreis, dem er seine Arbeitsergebnisse präsentieren muss, im Vergleich zu einer Expertenrolle 


textcolor{red}{hier Tabelle Untersuchungspopulation  einfügen  eigene Darstellung (als jpg Hinterlegen}

\subsection{Vorbereitung der Interviews und Durchführung}

Die vorbereitenden Tätigkeiten, erfolgten in parallelen Arbeitsschritten. Hierzu gehörten die Erstellung, Anpassung und Freigabe des Leitfadens. Neben dieser Tätigkeit erfolgte die erste Kontaktaufnahme bei potentiellen Interviewpartnern per persönlicher Vorstellung oder Skype-Anruf. In einem Fall wurde der Erstkontakt per Emailanfrage hergestellt. Bei der Vorstellung wurden die potentiellen Interviewpartner über die Masterarbeit, das Ziel und den Nutzen des Interviews ebenso wie die geplante Dauer informiert. Den Experten wurde in diesem Zusammenhang die Anonymisierung der Expertennamen in der Transkription zugesichert.

Mit dem Hinweis auf die Anonymisierung sollte im Interview eine offene und vertrauliche Gesprächsumgebung geschaffen werden. Ebenso ist in der Darstellung der Auswertung eine Zuordnung einzelner aussagen durch dritte nicht mehr möglich. Damit waren alle Interviewpartner einverstanden. 

Mit der Übergabe der Audioaufnahmen an den Lehrstuhl im Rahmen der Qualifikationsarbeitsabgabe waren neun Experten einverstanden, nachdem erklärt wurde, dass eine Veröffentlichung der Masterarbeit nicht vorgesehen ist. 

Da sich in den Vorgesprächen mit dem zehnten Experten herausgestellt hat, dass dieser die Nutzung von Modellierungssprachen sehr kritisch sieht, war das Interesse an dieser Perspektive mit Hinblick auf die Forschungsfrage sehr ausgeprägt. Nach Rücksprache mit dem Masterarbeitsbetreuer des Lehrstuhls, wurde das Experteninterview mit dem Experten, der der Übergabe der Audioaufnahme nicht zugestimmt hat, dennoch geführt. Die Audioaufnahme wurde, entsprechend der Vereinbarung mit diesem Experten, nicht dem Lehrstuhl übergeben. 

Mit der Zusage zum Interview, wurden die Einladungen verschickt, in diesen wurde erneut auf die Vertraulichkeit hingewiesen, auch wurde über das Thema des Interviews kurz informiert. Den Interviewteilnehmern wurde im Vorfeld nur bei explizitem Wunsch der Leitfaden zugeschickt. Damit sollte ein möglichst natürlicher Fragenverlauf und unvorbereitete Antworten generiert werden.

Lediglich ein Interviewteilnehmer wollte vorab den Leitfaden für das Interview erhalten, dieser wurde mit einer Vorlaufzeit von einer Woche dem Interviewpartner zugeschickt. Für die Interviews wurde das Videokonferenztool Zoom genutzt, in einem Fall musste Aufgrund von anbieterseitigen Serverproblemen auf Skype for Business zurückgegriffen werden. Für die zur Aufnahme der Interviews wurde die Software OBS-Studio verwendet. Die Interviews fanden zwischen dem 13.11.2025 und dem 19.12.2025 statt. Die Gesamtdauer der zehn Interviews betrug 8 Stunden 15 Minuten. 
Da es für das Transkribieren keine allgemein festgehaltenen Regeln gibt, ist die Aufstellung und Einhaltung eigener Transkriptionsregeln ein Teil der wissenschaftlichen Vorgehensweise.\footnote{J. Gläser, G. Laudel, Experteninterviews und Qualitative Inhaltsanalyse, VS Verlag 2010 S.193}

Aus forschungsökonomischen Gründen wird bei der Transkription auf zu detaillierte Transkriptionsregularik wie sie z. B. Von Kallmeyer und Schütze empfohlen werden, abgesehen. \footnote{Vgl. S. Fuß, U. Karbach, Grundlagen der Transkription: Eine praktische Einführung, Verlag Barbara Budrich GmbH, 2019, S.35} Die hier angewandten Transkriptionsregeln orientieren sich an den von Kuckartz aufgestellten Regeln. \footnote{Vgl. S. Fuß, U. Karbach, Grundlagen der Transkription: Eine praktische Einführung, Verlag Barbara Budrich GmbH, 2019, S.30 } 


\begin{enumerate}
    \item Es wird wörtlich transkribiert.
    \item Sprache und Interpunktion werden leicht geglättet, jedoch wird das Sprechen ``ohne Punkt und Komma`` nicht in kurze Sätze unterteilt.
    \item Deutlich längere Pausen werden durch Klammern (\dots) gekennzeichnet.
    \item Alle Angaben, die einen Rückschluss auf eine befragte Person erlauben, werden anonymisiert.
    \item Absätze der interviewenden Person und des befragten Experten werden klar voneinander abgehoben.
    \item Jeder Sprechbeitrag wird als eigener Absatz transkribiert, um die Lesbarkeit des Transkripts zu erhöhen.
    \item Relevante nonverbale Aktivitäten werden im Transkript festgehalten.
    \item Unverständliche Wörter werden durch \texttt{(unv.)} als solche kenntlich gemacht.
\end{enumerate}

Für die Umwandlung und Komprimierung der Audiodateien der VLC-Media-Player. Die Transkription wurde von der Software Whisper-Ai unterstützt. Hierzu wurden die mp3-Audiodatein von Whisper-Ai transkribiert., im Anschluss wurde das Transkript mit den Audiodateien verglichen und fehlerhafte Teile ausgetauscht. Ein häufig vorkommendes Beispiel für eine inkorrekte Transkription; die Software erkannte eine Textpassage als „DAV-Verbund“ dies wurde dann durch „DRV-Bund“ ersetzt, da die Interviewteilnehmer in diesem Kontext von ihrem Arbeitgeber der DRV-Bund gesprochen haben. 




\section{ Datenauswertung qualitatives induktives empirisches Forschungsdesign}

Die Auswertung von qualitativen Interviews wird mit der Methodik, die Mayring entworfen und als qualitative Inhaltsanalyse bezeichnete, durchgeführt. Mayring hat für die qualitative Inhaltsanalyse ein allgemeines Ablaufmodell entworfen. Diese Systematik, wie in Abbildung 3.1 dargestellt, wird im Folgenden näher erörtert, da Sie die Grundlage für die Auswertung der Experteninterviews darstellt. 

\begin{figure}[htbp]
    \centering
    \includegraphics[width=0.7\textwidth]{Abbildung2inhaltsanalytischesAuflaubmodell}
    \caption{Allgemeines inhaltsanalystisches Ablaufmodell S.61 aus P. Mayrings Qualitative Inhaltsanalyse}
    \label{fig:Abbildung2inhaltsanalytischesAuflaubmodell}
\end{figure}


Die Systematik beginn im ersten Schritt mit der Festlegung des Materials. In diesem Schritt wird die Auswahl festgelegt, „welches Material der Analyse zugrunde gelegt werden soll“\footnote {P. Mayring Qualtitative Inhaltsanalyse Grundlagen und Techniken, Beltz, 2015, S54}. Im zweiten Schritt, der Analyse der Entstehungssituation wird beschrieben wie, von wem und unter welchen Bedingungen das Material produziert wird und zu welchem Zweck. “\footnote {P. Mayring Qualtitative Inhaltsanalyse Grundlagen und Techniken, Beltz, 2015, S55}. Im Dritten Schritt werden die formalen Charakteristika des Materials beschrieben. Dies stellt in der Regel einen niedergeschriebenen Text dar. “\footnote {P. Mayring Qualtitative Inhaltsanalyse Grundlagen und Techniken, Beltz, 2015, S55}.

Der vierte Schritt Richtung der Analyse ist für die Klärung, unter welcher Perspektive das Datenmaterial analysiert werden soll, von den von Mayring aufgezeigten Möglichkeiten wird sich an dieser Stelle für die Beschreibung des im Material beschriebenen Gegenstandes entschieden. Die Analysepunkte, die sich mit dem psychischen Zustand des Kommunikationspartner oder der Vorgehensweise in der Literaturwissenschaft befassen, sind hinsichtlich der hier gestellten Forschungsfragen nicht geeignet. “\footnote {P. Mayring Qualtitative Inhaltsanalyse Grundlagen und Techniken, Beltz, 2015, S58} Gefolgt wird im nächsten Schritt, die Mayring Theoretische Differenzierung der Fragestellung nennt, das Textmaterial mit der bestehenden Theorie der Thematik in Verbindung gebracht. Es soll eine Verknüpfung der bisherigen Forschung zum Forschungsgegenstand erfolgen. “\footnote {P. Mayring Qualtitative Inhaltsanalyse Grundlagen und Techniken, Beltz, 2015, S59}.  

Im sechsten Schritt wird die passende Analysetechnik bestimmt und das konkrete Ablaufmodell festgelegt. Hier bestehen mehrere Grundformen , die miteinander kombinierbar sind.  \footnote { vgl. P. Mayring Qualtitative Inhaltsanalyse Grundlagen und Techniken, Beltz, 2015, S67}.   Hierzu zählen die Zusammenfassung. Die Explikation und die Strukturierung. Mithilfe der Zusammenfassung wird das Textmaterial mithilfe von Kürzen, Reduzieren und Paraphrasieren so reduziert, dass die wesentlichen Inhalte bestehen bleiben und das Datenmaterial an Übersichtlichkeit gewinnt. Die Explikation wird genutzt, wenn Textpassagen ergänzt werden müssen, um das Verständnis derselbigen zu verdeutlichen. Mit der Strukturierung als dritte Analysetechnik, besteht die Möglichkeit der Kategorienbildung und der Einordnung des vorliegenenden Material. \footnote { vgl. P. Mayring Qualtitative Inhaltsanalyse Grundlagen und Techniken, Beltz, 2015, S67}.

Im siebten Schritt des Vorgehensmodells werden die Analyseeinheiten definiert, also die größe der zu Analysierenden Textpassagen und die Auswertungsreihenfolge wird festgelegt. Im 8 Schritt werden die Analyseschritte gemmäß des Ablaufmodells und die Rücküberprüfung des Kategoriensystems vorgenommen. Die Rücküberprüfung stellt sicher, dass die zusammengefassten Gewonnen aussagen noch das Ausgangsmaterial repräsentieren. Im neunten Schritt werden nun die Ergebnisse Zusammengestellt und unter Bezugname auf die Forschungsfragen interpretiert.\footnote {. vgl. P. Mayring Qualtitative Inhaltsanalyse Grundlagen und Techniken, Beltz, 2015, S61}
Im, abschließende Schritt erfolgt die Anwendung der Gütekriterien Reliabiltät und Validität eine Einschätzung der Analyse auf Tauglichkeit  footnote {. vgl. P. Mayring Qualtitative Inhaltsanalyse Grundlagen und Techniken, Beltz, 2015, S.123f}

Die Analysetechnik „Zusammenfassung“ wurde für die Auswertung der Experteninterviews gewählt. Um das oben erwähnte Ziel der Zusammenfassung zu erreichen hat Mayring ebenfalls eine Vorgehensweise festgelegt, die für diese Inhaltsanalyse umgesetzt wird. Es handelt sich um folgendes siebenstufige Vorgehen:

1.Stufe: Die Analyseeinheiten werden bestimmt

2. Stufe: Die Inhaltstragendenden Textstellen werden regelgeleitet paraphrasiert

3. Stufe: Das Angestrebte Abstraktionsniveau wird bestimmt. Die Paraphrasen werdne unter diesem Abstraktionsniveau generalisiert

4. Stufe: erste Reduktion durch Auswahl und Streichung von bedeutungslgeichen Paraphrasen

5 Stufe: wzeite Reduktion durch Bündelung, Konstruktion und Integration auf dem angestrebten Abstaktionsniveau

6. Stufe: Zusammenstellung der Aussagen als Kategoriesystem

7. Stufe: Rücküberprüfung des zusammenfassenden Kategoriesystems am Ausgangsmaterial.
 
Auf Basis der Antworten der Experten konnten folgende Hauptkategorien definiert werden.

textcolor{red}{ Hier Tabelle einfügen mit den Hauptkategorien } 

Es folgt die Darstellung der Ergebnisse der Experteninterviews nach der oben vorgestellten Methodik.






\ 
