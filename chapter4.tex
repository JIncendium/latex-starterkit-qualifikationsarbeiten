% !TEX root = main.tex
% !TEX TS-program = pdflatex
% !BIB program = biber


\lstset{language=C}

\chapter{Ergebnisse gemäß der Methodik Vorgehensmodells}


%
%
%Hier wird die Hauptarbeit entstehen
%
%
%Noch einmal prüfen ob der Interviewleitfaden hier passt ?

\section{Interviewleitfaden}

Um die offenen Fragen in den Gesprächen mit den Softwarearchitekten zu strukturieren, ergibt sich die Notwendigkeit eines Leitfaden-Einsatzes. Hierdurch soll gleichzeitig auch eine gewisse Vergleichbarkeit der Interviews geschaffen werden.\footnote{Helfferich, Cornelia in Handbuch der Empirischen Sozialforschung, Springer VS, 2022, S. 875-892}

Die Interviewfragen werden in Orientierung an den Phasen eines wissenschaftlichen Interviews (Einleitung, Hauptteil und Abschluss) sowie gleichzeitig nach inhaltlichen Bereichen sortiert. 
Die erste Version des Leitfadens soll nach einem oder zwei Pretests überarbeitet werden. Der Pretest soll Erkenntnisse über die Dauer des Interviews, die Sinnhaftigkeit der Fragen-Reihenfolge, Verständlichkeit der Fragen und potentielle Fehlerquellen in der Interviewdurchführung ergeben.

Der einleitende Teil des Leitfadens erfragt allgemein nach der Arbeit des Softwarearchitekten, mit welchen Werkzeugen (Modellierungssprachen) gearbeitet wird und ab wann Modellierung notwendig wird. Die im Einleitungsteil enthaltenen Fragen sollen erzählgenerierende Wirkung entfalten. Die Aufforderung über die Tätigkeit zu sprechen folgt der Empfehlung von Helfferich \footnote{Helfferich, Cornelia in Handbuch der Empirischen Sozialforschung, Springer VS, 2022, S. 884}Damit sind auch Fragen nach der Dauer der Tätigkeit und der fachlichen Ausrichtung eingeschlossen.

Die Fragen des Hauptteiles fokussieren sich in einem ersten Abschnitt mit ihren jeweiligen Formulierungen auf das Begriffsverständnis von Softwarearchitektur, Modellierungssprache und Anforderungen. Die Leitfrage nach den wichtigen Anforderungen an Modellierungssprachen und die folgefragen zur Praxisrelevanz und den von Architekten genannten Anforderungen, soll die erste Forschungsfrage beleuchten.

Der zweite Abschnitt des Hauptteils widmet sich der zweiten Forschungsfrage. Hier wird durch die Frage nach den Herausforderungen bei der Nutzung bestehender Modellierungsfragen und den Missverständnissen noch einmal der konkrete Blick des Architekten auf die Modellierungssprachen erfragt. Die Schärfung des Kontextes erfolgt über die Nachfragen nach spezifischen Verbesserungspotentialen in Bezug auf Semantik, Notation und Toolunterstützung. Mit der Frage nach speziellen Anwendungsfällen, in denen die bestehenden Sprachen Probleme haben wird, versucht den Blick auf die Anforderungen zu verbessern. Bevor zuletzt die offene Frage nach Verbesserungspotentialen gestellt wird.

Der Schlussteil des Leitfadens wird von der Frage nach der perfekten Modellierungssprache geprägt. Dies gibt dem Softwarearchitekten die Möglichkeit, bereits Gesagtes zusammenzufassen, zu ergänzen oder zu ersetzen und bildet somit eine offene Reaktionsmöglichkeit des Interviewten. Der Leitfaden dient darüber hinaus dem Forschenden zur thematischen Fokussierung und der Organisation des eigenen Wissens. Der Leitfaden wird vollständig im Anhang abgebildet. 

\section{Aussuchen der Zielgruppe/der Interviewpartner}
Um Softwarearchitekten für die Interviews zu finden, wird folgendermaßen vorgegangen. Der Forschende hat durch seine Tätigkeit im Rechenzentrum der DRV-Bund Zugang zum Architekturboard des Rechenzentrums. Die dort tätigen Softwarearchitekten werden zunächst mit einem Aufruf zur Teilnahme durch ein persönliches Gespräch gebeten. Da zu erwarten ist, dass die notwendige Anzahl von Interviewpartnern auf diese Weise noch nicht erreicht wird, wird parallel ein Anschreiben per Email an Softwarearchitekten verschickt. 	

Ziel ist es, so viele Interviews durchzuführen, bis eine Sättigung an neuen Informationen erreicht ist. Ob dieses Ziel in der Forschungspraxis erreicht werden kann, ist zum jetzigen Zeitpunkt offen und mit Hinblick auf den zeitlichen Rahmen der Qualifikationsarbeit fraglich. Allerdings liegt der Fokus zunächst auf der Erhebung der Anforderungen von Softwarearchitekten. Da es sich bei den Antworten der Interviewpartner nicht um einen Anspruch auf Generalisierbarkeit oder auf Übertragbarkeit auf die zugrunde liegende Grundgesamtheit handeln soll, sondern um Personen die mit ihrem spezifischen Wissen für die Beantwortung der Forschungsfragen besonders relevant sind, ist eine zweckorientierte Stichprobe fachlich sinnvoll. Geeignete Auswahlkriterien für die Interviewpartner sind Position im Unternehmen, Berufserfahrung und Fachwissen.

\section{Inhaltsanalyse}
