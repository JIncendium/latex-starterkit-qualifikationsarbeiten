% !TEX root = main.tex
% !TEX TS-program = pdflatex
% !BIB program = biber

\lstset{language=C}





%\section{Aussuchen der Zielgruppe/der Interviewpartner}
%Um Softwarearchitekten für die Interviews zu finden, wird folgendermaßen vorgegangen. Der %Interviewer hat durch seine Tätigkeit im Rechenzentrum der DRV-Bund Zugang zum Architekturboard des Rechenzentrums in Würzburg und zu einem weiteren Architektur-Board einer IT-Abteilung in Berlin. Von den dort tätigen Softwarearchitekten werden zunächst potentielle Interviewkandidaten angefragt. Um verschiedene Blickwinkel auf die Forschungsfragen zu erhalten, werden die Interviewpartner aus verschiedenen Hierarchiebenen und Tätigkeitsfeldern in der Softwarearchitektur gewählt. Darüber hinaus werden sowohl interne Softwarearchitekten als auch externe Mitarbeiter in den Architekturboards zum Interview eingeladen. Mit einem Skype-Anruf wird erfolgt die erste Kontaktaufnahme. In dieser werden die Forschungsfragen vorgestellt und um ein Interview gebeten. Bei Zustimmung wird die Intervieweinladung per Email verschickt.
%Zehn Softwarearchitekten konnten auf diesem Weg für ein Interview gewonnen werden, hiervon musste einem der Fragebogen vorab zugeschickt werden, sonst wäre es nicht zu einer Teilnahme gekommen.  


%Ziel ist es, so viele Interviews durchzuführen, bis eine Sättigung an neuen Informationen erreicht ist. Ob dieses Ziel in der Forschungspraxis erreicht werden kann, ist allerdings zu hinterfragen, ob dieses Ziel  Hinblick auf den zeitlichen Rahmen der Qualifikationsarbeit erfüllt werden kann. Allerdings liegt der Fokus zunächst auf der Erhebung der Anforderungen von Softwarearchitekten. Da es sich bei den Antworten der Interviewpartner nicht um einen Anspruch auf Generalisierbarkeit oder auf Übertragbarkeit auf die zugrunde liegende Grundgesamtheit handeln soll, sondern um Personen die mit ihrem spezifischen Wissen für die Beantwortung der Forschungsfragen besonders relevant sind, ist eine zweckorientierte Stichprobe fachlich sinnvoll. Geeignete Auswahlkriterien für die Interviewpartner sind Position im Unternehmen, Berufserfahrung und Fachwissen, dies wurde druch die oben genannten Kriterien erfüllt.



\chapter{Ergebnisse der Experteninterviews mit Softwarearchitekten gemäß der Methodik }

Der folgende Abschnitt widmet sich den Ergebnissen der Experteninterviews und stellt diese vor. 
Mit Hilfe der induktiven inhaltsanalytischen Auswertung der Transkripte wurden drei Hauptkategorien gebildet. Diese wiederum gliedern sich weiter in insgesamt 18 Unterkategorien. Die folgende Tabelle
\ref{tab:architektur} stellt die Kategorien dar mit Angabe der Anzahl der Fundstellen sowie der Angabe der Anzahl, wie viele der Architekten sich hierzu geäußert haben. 


\begin{table}[htbp]
    \centering
    \includegraphics[width=0.9\textwidth]{TabelleHauptkategorien.png}
    \caption{Hauptkategorien und Kategorien zur Beantwortung der Forschungsfragen Systemarchitektur als Übersicht}
    \label{tab:architektur}
\end{table}



Die vollständigen Tabellen zu den Fundstellen in den Interviews, die Grundlage für die inhaltsanalytische Auswertung sowie die vollständigen Tabellen für die Unterkategorien finden Sie im Anhang. 


Die Hauptkategorien teilen sich in die Tätigkeit der Softwarearchitekten, die Anforderungen an Modellierungssprachen und schließlich die Herausforderungen und Verbesserungspotentiale. In der Kategorie A werden die Angaben der Architekten zu ihrer Sicht auf Modellierung, ihre Arbeitsweise, Nutzung von Werkzeugen bzw. Modellierungssprachen aufgeführt.  In der Kategorie A7 Varia finden sich Aussagen der Architekten, die exkursartigen Charakter aufweisen z.B. Ansichten der Softwarearchitekten über die Zukunft der Softwarearchitektur oder Anmerkungen so Architekturkriterien.
Die Kategorie B und C befassen sich explizit mit den Aussagen, die direkt den Forschungsfragen zugeordnet werden können. In Kategorie B werden die Antworten zu den Anforderungen an Modellierungssprachen gezeigt. Die Kategorie C widmet sich den Herausforderungen und Verbesserungspotentialen, die die Architekten aufzeigen.  
Auf die Ergebnisse der einzelnen Kategorien wird im nächsten Abschnitt eingegangen.




%Hier wird die Hauptarbeit entstehen


\section{Auswertung der offenen Fragen - Tätigkeit eines Softwarearchitekten}

\subsection{Aufgabenfeld des Architekten}

Die Antworten zu den Tätigkeiten sind in sieben Unterkategorien gegliedert. Die erste  Unterkategorie A1 Aufgabenfeld des Architekten, zeigt anhand der Antworten der Softwarearchitekten die breite des Spektrums an Aufgaben. Kommunikation[JB3.1] ist, der am häufigsten verwendete Begriff in diesem Zusammenhang. 
Als Ankerbeispiel hierzu kann aus dem vierten Interview folgendes Zitat dienen: 

\begin{quote}
\textit{Softwarearchitekt 4: „wo es dann auch hieß, naja, wie bist du Softwarearchitekt geworden? Naja, irgendwann mal vom Entwickler da falsch abgerutscht, zu viel kommuniziert und zack, Softwarearchitekt.“}

\hfill \textit{(Interview 4, S.11 )}
\end{quote}

Von den Architekten wird an vielen Stellen in den Interviews verdeutlicht, dass die Kommunikation der Architekturvision und der Modellierung im Vordergrund der Tätigkeit stehen. Ebenso Kommunikation mit Kunden und Kollegen mit der dazugehörigen Erklärung der Modellierung und ihrer Ziele. Die Schaffung eines gemeinsamen Verständnisses und einer gemeinsamen Sprache zur Problembeschreibung zwischen den Fachabteilungen, den Betriebsabteilungen und dem Management fällt auch in den Kommunikationsbereich.
Koordinierende Aufgaben wie die Tätigkeit in den Architekturboards oder die Akquise der internen und externen Softwarearchitekten gehören auch zum Tätigkeitsbereich. Je nach Hierarchieebene des jeweiligen Softwarearchitekten gehört entweder die Vorgabe der Rahmenbedingungen, Auswahl der Standards und Technologien oder die Umsetzung dieser Punkte zum Arbeitsprofil. 
Ein weiterer zentraler Aspekt ist die Modellierung, die Pflege dieser Modelle und die Dokumentation unter Berücksichtigung unterschiedlicher Architekturebenen. Dies erfasst das Einbeziehen der Betriebsarchitektur und der Softwarearchitektur. An dieser Stelle wird auch von einzelnen Softwarearchitekten betont, dass die Beherrschung der korrekten Notation und vergangene Praxiserfahrung als Entwickler als zwingende Voraussetzungen erachtet würden um das Profil eines guten Architekten zu erfüllen. Gerade die Praxiserfahrung hilft den Architekten bei der Antizipation von Änderungen in der Implementierungsphase. Auch die Wahl der richtigen Modellierungssprache kann erst mit genügend Praxiserfahrung getroffen werden.

Die nächste Unterkategorie A2 zeigt auf, welche Modellierungssprachen aktuell von den Softwarearchitekten benutzt werden. Die Ergebnisse werden kurz in der folgenden Abbildung dargestellt.

\subsection{Genutzte Modellierungssprachen}

Die nächste Unterkategorie A2 zeigt auf, welche Modellierungssprachen aktuell von den Softwarearchitekten benutzt werden. Die Ergebnisse werden kurz in der folgenden Abbildung dargestellt.

\begin{figure}[htbp]
    \centering
    \includegraphics[width=0.9\textwidth]{Kaptitel_4 _Abb_Genutzte_Modellierungssprachen.png}
    \caption{Genutzte Modellierungssprachen (eigene Darstellung)}
    \label{fig:genutzteModellierungssprachen}
\end{figure}


Die befragten Softwarearchitekten nutzen demnach überwiegend mehr als eine Modellierungssprache. Ein Softwarearchitekt hingegen verzichtet bewusst auf die Nutzung einer formalisierten Modellierungssprache. Die übrigen neun Softwarearchitekten benutzen alle UML, sieben Verwenden BPMN 2.0. Die Modellierungssprachen Archimate und ERM werden in geringerem Umfang verwendet. Lediglich zwei Softwarearchitekten nutzen ausschließlich UML.\footnote{siehe Anhang A2 Modellierungssprachen in Nutzung}

\subsection{Modellierungsbeginn und Modellierungsende}

Die Unterkategorie A3 befasst sich mit den Definitionen von Modellierungsbeginn und Modellierungsende der Softwarearchitekten. 
Die Antworten zum Beginn zeigen, dass die Architekten unterschiedliche Tätigkeiten mit dem Beginn verbinden, wie z.B. das erste informelle Design auf dem Papier oder bereits die geistige Tätigkeit vor dem ersten Verschriftlichen.  
Große Einigkeit besteht bezüglich der Voraussetzungen, die erfüllt sein müssen, bevor man beginnen kann. So wurde von den Architekten festgehalten, dass die Problemformulierung, das Bestehen von fachlichen Anforderungen oder der Eingang einer Anforderung den Beginn der Modellierung markieren. Auch die Problemanalyse, die Beschreibung des Problems und erste Komplexitätsverringerung der Anforderungen wurden von den Interviewteilnehmern genannt. 
Im Gegensatz dazu wurde das Ende der Modellierung sehr unterschiedlich definiert. Die Übergabe des Modells an die Entwicklung war hier eine Antwort. Auch das Fertigstellen der graphischen Modellierung sowie die Produktivsetzung der Software mit anschließendem Abgleich des Modells an die Implementierung wurden als Ende der Modellierung genannt.\footnote{siehe Anhang A3 Modellierungsbeginn, Modellierungsende}

\subsection{Genutze Werkzeuge}

Die Frage nach den Werkzeugen, die Softwarearchitekten für ihre Tätigkeitsausübung verwenden wurde in eine weitere Unterkategorie eingeordnet. Die folgende Tabelle zeigt die Werkzeuge auf, welche die Softwarearchitekten zum Modellieren verwenden.

\begin{table}[h]
\centering
\caption{Genutzte Werkzeuge für das Modellieren (nach Häufigkeit)}
\label{tab:modellierungswerkzeuge}
\begin{tabular}{l c}
\toprule
\textbf{Werkzeug} & \textbf{Anzahl Nennungen} \\
\midrule
Draw.io & 5 \\
Microsoft Visio & 5 \\
Adonis & 2 \\
Camunda Modeller & 2 \\
Flipchart & 2 \\
Visual Paradigm & 2 \\
IntelliJ mit Plugins & 1 \\
Microsoft Word & 1 \\
Papier & 1 \\
Umodel & 1 \\
\bottomrule
\end{tabular}
\end{table}

Die qualitative Auswertung zeigt auch, dass die Softwarearchitekten überwiegend zwei oder mehr Werkzeuge für ihre Modellierungstätigkeiten verwenden /footnote A4 Tools in Nutzung. Die Nutzung von Werkzeugen, die lediglich eine einfache grafische Darstellung ermöglichen macht 61\% der Angaben aus. Eine spezielle Modellierungssoftware, wie der Camunda Modeller oder UModel werden nur von weniger als der Hälfte der Architekten verwendet.


\subsection{Herangehensweise bei der Modellierung}

Die Unterkategorie mit den meisten Fundstellen in den 10 Interviews befasst sich mit den Antworten der Architekten zur Kategorie A5 Herangehensweise bei der Modellierung. Hier sind mithilfe der qualitativen Inhaltsanalyse sechs Schwerpunkte herausgearbeitet worden.

\newcounter{punkt} % Zähler für die Liste
\begin{list}{\arabic{punkt}.}{\usecounter{punkt}\setlength{\leftmargin}{2cm}}
    \item Rolle und Zweck von Modellierung
    \item Vorgehen und Arbeitsweise
    \item Modellierungssprachen, Standards und Formalität
    \item Kommunikation und Anspruchsgruppen
    \item Visualisierung und Darstellungsformen
    \item Qualitätssicherung
\end{list}

Zum ersten Punkt, die Rolle des Architekten findet sich neben der Aussage, dass Modellierung grundsätzlich notwendig ist auch die Überzeugung, dass auf Modellierung bei kleinen Anwendungen verzichtet werden kann. Die Modellierungstätigkeit soll eine Arbeitserleichterung für nachgelagerte Prozesse sein und der Strukturierung von Gedanken, Entscheidungen und Kommunikation dienen.
Auch beim zweiten Punkt Vorgehensweise kann festgestellt werden, dass Architekten sehr unterschiedlich an Probleme herangehen. Hier stehen die Aussagen, dass zur Entscheidungsfindung am besten zwei Lösungen modelliert werden sollten neben Aussagen, dass Modellierung erfahrungsbasiert ohne Muster abläuft.  Ein Architekt betont das Mitdenken von Ressourcenengpässen in seiner Vorgehensweise. 

Ein Ausschnitt aus der inhaltsanalytischen Auswertung der Aussagen zur Unterkategorie A5, Herangehensweise bei der Modellierung, findet sich auf der folgenden Seite in Tabelle \ref{tab:A5 Ausschnitt}

\begin{landscape} % Querformat-Seite
\begin{table}[H]
    \centering
    % Caption über der PDF
    \caption{Ausschnitt der Zusammenfassung A5}
    \label{tab:A5 Ausschnitt}
    \vspace{1em} % Abstand zwischen Caption und PDF
    % PDF einfügen, etwas kleiner
    \includegraphics[width=1.5\textheight]{Kapitel_4_A5_Ausschnitt.pdf}
\end{table}
\end{landscape}

 
Auch im Punkt Modellierungssprachen, Standards und Formalität stehen sich gegensätzliche Aussagen gegenüber. Softwarearchitekt 7 vertritt die Position, dass Modellierung ohne Modellierungssprache und ohne formale Standards möglich, aber umständlich ist. Folgendes Ankerbeispiel des Interviews mit Softwarearchitekt 7 soll dies verdeutlichen.

\begin{quote}
\textit{Interviewer: „Würdest du sagen, du könntest auch ohne eine formale Modellierungssprache modellieren und auch gut und robust modellieren?“}

\textit{Softwarearchitekt 7: „Ja, was das Modellieren kann, kann man natürlich auch ohne Standards oder ohne so etwas. Aber dann ist immer das große Problem das babylonische Sprachgewirr […] Und dann muss man aber das, was man sich dabei denkt, dann immer auch in der Legende aufnehmen, mühsam. Und der andere Partner, mit dem man das dann durchgeht, der muss das erst mal verstehen, welche Notation ich mir jetzt habe einfallen lassen. Also im Prinzip ja, das würde aber bedeuten, dass man das Rad eben x-fach neu erfindet.“}

\hfill \textit{(Interview 7, S.3)}
\end{quote}

Im Kontrast hierzu steht die Aussage von Softwarearchitekt 8 auf die gleiche Frage: 

 
 \begin{quote}

\textit{Softwarearchitekt 8:„Nee, das geht gar nicht, weil gewisse Regeln musst du einhalten. […] wenn es nicht in irgendwelchen Modellierungssprachen mit wirklich hinterlegt ist und jeder frei rum malen könnte, dann würde ich sagen, haben wir einen Surrealismus, […] bringt mir nichts.“}

\hfill \textit{(Interview 8, S.7)}
\end{quote}

Im nächsten Punkt bezüglich der Kommunikation mit den unterschiedlichen Anspruchsgruppen ist die Mehrheit der befragten Softwarearchitekten der Überzeugung, dass diese eine besonderen Stellenwert hat und eine enge Abstimmung erfordert, gerade weil die Architekten oftmals keinen direkten Einfluss mehr auf die Entwicklung nehmen, sondern eher Rahmenbedingungen schaffen. Auch wird betont, dass Konfliktfähigkeit gerade im Zusammenhang mit dem Kunden (extern oder intern) sehr hilfreich ist.
Bei der Visualisierung setzen die Architekten auf eine Kombination aus graphischer Darstellung und erklärenden Texten, um die Verständlichkeit zu verbessern.
Bei Qualitätssicherung ihrer Modellierung, haben die befragten Softwarearchitekten unterschiedliche Auffassungen. Auf der einen Seite wird Qualitätssicherung nur für Modellierung nicht für notwendig befunden, auf der anderen Seite werden Qualitätssicherungs-Prüfpunkte (QS-Gates) unter einbeziehen der zuständigen Softwarearchitekten empfohlen. Diese überprüfen an diesen Punkten, ob die Entwicklung mit der Architekturmodellierung konsistent ist.

\subsection{Fortbildungskontext}

Die Ergebnisse zu der Frage liefert sehr unterschiedliche Antworten. Die Softwarearchitekten lernen zum einen über die Fortbildungsangebote ihres Arbeitgebers. Hier liegt der Fokus auf neuen Technologien oder den spezifischen Arbeitskontext, dies kann auch mit dem Erwerb von Zertifizierungen einhergehen. Darüber hinaus wird über das praktische Anwenden mit Modellierungssprachen der Umgang mit ihnen geübt. Weiter genannte Fortbildungsinstrumente sind die Internetrecherche einschlägiger Foren und Firmenhomepages, der Austausch mit Kollegen, Emailverteiler, Fachlektüre sowie Architekturpodcasts.\footnote{siehe Anhang A6 Fortbildungskontext}

\subsection{Sonstiges}

In dieser Kategorie finden sich Aussagen der Architekten zur Einschätzung der Zukunft der Architektur, aber auch grundlegende Ansichten zum Architekturverständnis bis hin zu Definitionen der einzelnen Softwarearchitekten zum Begriff Softwarearchitektur. So sind Definition der Softwarearchitektur, Softwarearchitektur ist der Bauplan, die Definition während die Entwicklung die Realisierung umsetzt, bei fließenden Grenzen, genannte Szenarien. 
Die Zukunft der Softwarearchitektur wird vielen Softwarearchitekten als komplexer und feingranularer eingeschätzt.\footnote{siehe Anhang A7 Varia} 

\section{Anforderungen an Modellierungssprachen}

Die Kategorie B Anforderungen an Modellierungssprachen gliedert sich wie in Tabelle \ref{tab:architektur} dargestellt.
In fünf Bereichen werden die Eigenschaften von Modellierungssprachen, die den Softwarearchitekten wichtig sind, zunächst aufgeführt. In einem weiteren Schritt werden die tätigkeitsspezifischen Anforderungen der Architekten dargestellt. Die drei darauffolgenden Abschnitte gehen auf die formalen, anwenderbezogenen und anwendungsbezogenen Anforderungen ein, die in den Interviews von den Softwarearchitekten genannt wurden.

\subsection{Eigenschaften von Modellierungssprachen}

Die Antworten der Softwarearchitekten hinsichtlich der Eigenschaften von Modellierungssprachen zeigen, dass viele Aspekte für die Architekten von Bedeutung sind. Es werden Standardisierung und Offenheit genannt. So sehen die Architekten UML als sehr gute Modellierungssprache an, weil diese bereits eine weite Verbreitung hat und den Standard vorgibt. Dies führt zur Vereinfachung der Kommunikation unter den Anwendern.
Die Modellierungssprachen sollen die Verständlichkeit verbessern und die Balance zwischen Komplexität und Verständlichkeit finden. Ein weiterer wichtiger Punkt ist die Fähigkeit der Modellierungssprache gut graphisch abgebildet zu werden. Darüber hinaus sollen sie eine gute Strukturierung vorgeben und unterschiedliche Sichten ermöglichen. Auch hier wird UML wieder als Beispiel für eine gute Modellierungssprache erwähnt.
Ein Architekt nimmt Bezug auf die Meta-Modelle der Modellierungssprachen und betont, dass es wichtiger ist die Anwendung des Meta-Modelles zu beherrschen als ein vertieftes Verständnis des Meta-Modells einer Modellierungssprache selbst. Bei komplexen Metamodellen ist eine Anleitung zur Nutzung notwendig.

\begin{quote}
\textit{Softwarearchitekt 3:„Ich muss die Anwendung des Meta-Modells verstehen. Und das erwarte ich von einer Modellierungssprache,[…] Und von einer Modellierungssprache erwarte ich, dass sie nutzbar ist, also Nutzbarkeit, Anforderung, ja. Selbst wenn sie komplexe Meta-Modelle haben, sozusagen, muss eine Anleitung geben, wie ich das nutze.“}

\hfill \textit{(Interview 3, S.4)}
\end{quote}

\subsection{Tätigkeitsspezifische Anforderungen}

Bei den tätigkeitsspezifischen Anforderungen handelt es sich um Anforderungen, welche die Softwarearchitekten formuliert haben, die mit Modellierung und Modellierungssprachen in Verbindung stehen, sich jedoch nicht in direkt den Definitionsbereich Anforderungen an Modellierungssprachen einordnen lassen.

So ergeben die Antworten der Softwarearchitekten, dass das Wissen und das Können im Umgang mit Modellierungssprachen zwingend vorhanden sein sollte. Auch ist die Nutzung einer etablierten Sprache genannt, um Kommunikationshürden abzubauen. Ebenfalls sind die Architekten der Ansicht, dass die Wahl der richtigen Diagrammart für die richtige Zielgruppe notwendig ist. 
Auch diese unterstützt die Anforderung, dass die Adressaten die Modellierung und die Intention des Architekten verstehen. Ein weiterer genannter Aspekt ist die Anforderung, an den Architekten selbst, die Modellierungstiefe so zu gestalten, dass jeder Adressat die richtige Abstraktionsebene einer Modellierung erhält: je mehr technisches Fachwissen vorhanden ist, desto höher die Detailtiefe. Ein Beispiel, dass hier genannt wurde ist, dass die Entwicklung sehr detaillierte Modellierung benötigt. In diesem Zusammenhang wird erwähnt, dass die zielgruppengerechte Darstellung wichtiger sei, als die formale Korrektheit. Das mehrfach genannte Beispiel hierfür ist die Aufbereitung der Modellierung gegenüber dem Management.
Das Ankerbeispiel hierzu aus dem achten Interview lautet: 

\begin{quote}
\textit{Softwarearchitekt 8:„Also je nachdem, wen du dort gegenüber hast, musst du einschätzen können, kann er schon so weit abstrahieren, so technische Sachen schon abstrahieren oder musst du das irgendwie noch in einer bildlichen Sprache? Also manchmal sehe ich mich da drin, wenn ich es meinen Kindern erklären kann, wenn die es kapiert haben, also mit Bildern und Händen und Füßen irgendwie mit. Ab dann ist es super, das ist das Minimum, was ich für das Management noch brauche irgendwie mit.“}

\hfill \textit{(Interview 8, S.3)}
\end{quote}

Eine weitere genannte Anforderung betrifft die Formalitäten um die Modellierung herum. Auch die schriftliche Fixierung der Anforderungen wird auch als wichtiger Aspekt genannt.

\subsection{Formale Anforderungen}

Die formalen Anforderungen der Softwarearchitekten an Modellierungssprachen können folgendermaßen zusammengefasst werden. 
Ein wesentliches Kriterium ist die Redundanzfreiheit, sodass Sachverhalte ohne Mehrfachabbildungen dargestellt werden können und Modelle konsistent sowie übersichtlich bleiben. In diesem Zusammenhang sollte für jede Modellierungsaufgabe eine eindeutig definierte und standardkonforme Lösung existieren, um eine einheitliche Anwendung der Sprache zu gewährleisten. Ebenso ist die Wartbarkeit von Modellen von großer Bedeutung. Modelle sollten so strukturiert und aufgebaut sein, dass sie langfristig verständlich bleiben und mit vertretbarem Aufwand angepasst oder erweitert werden können.
Darüber hinaus sollte eine Modellierungssprache die Abbildung rekursiver bzw. hierarchisch verschachtelter Strukturen ermöglichen, um komplexe Systeme auf unterschiedlichen Abstraktionsebenen angemessen darzustellen. Ein weiteres zentrales Qualitätsmerkmal ist die Eindeutigkeit der Semantik, damit Modelle von verschiedenen Beteiligten konsistent interpretiert werden und Missverständnisse vermieden werden. Hinzu kommt die Einfachheit des Metamodells der Modellierungssprache.
Neben strukturellen und semantischen Anforderungen spielt auch die Übersichtlichkeit eine wichtige Rolle. Die Sprache sollte die Darstellung komplexer Zusammenhänge in klarer Form unterstützen und unterschiedliche Sichten auf ein System ermöglichen. Schließlich sind Einfachheit und Verständlichkeit entscheidende Faktoren, insbesondere im Hinblick auf den Einsatz von Modellen zur Dokumentation und zur Kommunikation zwischen verschiedenen Stakeholdern. \footnote{vgl. Anhang B3 Formale Anforderungen}

\subsection{Anwenderbezogene Anforderungen}

Die Softwarearchitekten nennen folgende Punkte, die für Sie im Hinblick auf anwenderbezogene Anforderungen von Bedeutung sind.
An Modellierungssprachen werden verschiedene Anforderungen gestellt, die sich insbesondere an den Bedürfnissen der Anwender und dem jeweiligen Einsatzzweck orientieren. Ein zentrales Kriterium ist die Einfachheit sowie eine gute Erlernbarkeit, verbunden mit einer niedrigen Einstiegshürde. Darüber hinaus sollte die Sprache auf einem offenen Standard basieren und keine proprietären Abhängigkeiten aufweisen. Gleichzeitig ist eine hohe Standardisierung wichtig, um eine einheitliche Anwendung sowie eine breite Verständlichkeit für viele Anwender zu gewährleisten.
Die Modellierung soll zweckabhängig sein und muss sich an den jeweiligen Adressaten orientieren. Dabei sind anwenderbezogene Anforderungen von zentraler Bedeutung. Die Modelle sollen so gestaltet sein, dass die Zielgruppen sowohl die Inhalte als auch die Intention der Architekten nachvollziehen können. 
Für einen Architekten ist die Verständlichkeit für die Adressaten ist dabei wichtiger als eine rein formal korrekte Darstellung.
Andere Architekten betonen hingegen, um unterschiedlichen Stakeholdern gerecht zu werden, sollte ein anpassbarer Detaillierungsgrad unterstützt werden. Zudem sind verschiedene, aufeinander abgestimmte Sichten erforderlich, etwa eine Managementsicht, die Sachverhalte vereinfacht darstellt, sowie Facharchitektur- und Entwicklersichten, die voneinander abstrahiert werden können und dennoch konsistent bleiben sollen. Eine zielgruppengerechte Modellierung sowie die Auswahl der jeweils passenden Diagrammart tragen wesentlich zur Verständlichkeit bei.
Folgende Punkte wurden ebenfalls genannt, Usability, Klarheit und eine gute Lesbarkeit der Modelle. Die Eindeutigkeit der Modelle stellt auch eine grundlegende Anforderung dar, um Fehlinterpretationen zu vermeiden.
Mehrfach wurde auch die Toolunterstützung als wesentliche Anforderung an eine Modellierungssprache genannt.\footnote{Anhang B4 Anwenderbezogene Anforderungen}

\subsection{Anwendungsbezogene Anforderungen}

Für die anwendungsbezogenen Anforderungen wird von den befragtem Softwarearchitekten die Operationalisierbarkeit genannt. Explizit wurde hierbei die Darstellung von zeitlichen Verläufen herausgestellt. Auch Mächtigkeit und Angemessenheit werden betont. Die Möglichkeit alle Sachverhalte im geforderten Detaillierungsgrad darzustellen wird als Anforderung formuliert. 
Ein Softwarearchitekt formuliert es gegenläufig, hier wird Ausdruckstärke als weniger wichtig angegeben. \footnote{B5 Anwendungsbezogene Anforderungen}

\section{Herausforderungen und Verbesserungspotentiale}

Der folgende Abschnitt stellt die Ergebnisse der qualitativen Inhaltsanalyse zu dem Schwerpunkt Herausforderungen und Verbesserungspotentiale.

\subsection{Limitationen von Modellierungssprachen} 

Die qualitative Auswertung der Interviews bringt folgende Ergebnisse hinsichtlich der bestehenden Limitationen von Modellierungssprachen.
Hierbei werden zunächst methodische und fachliche Grenzen beschrieben. Nach Einschätzung der Befragten lassen sich bestimmte Inhalte, wie beispielsweise Aspekte der Betriebsarchitektur mit gängigen Modellierungssprachen nicht oder nur eingeschränkt abbilden. Zudem wird darauf hingewiesen, dass UML zur Darstellung von Anwendungslandschaften als wenig geeignet wahrgenommen wird und BPMN Grenzen bei der Abbildung organisatorischer Verantwortlichkeiten aufweist. Darüber hinaus wird angegeben, dass Modellierungssprachen bei betrieblichen bzw. operativen Anforderungen nur eingeschränkt eingesetzt werden können. Ebenso werden sie als ungeeignet für Fragestellungen beschrieben, die ein Ranking von Merkmalen oder die Abwägung von Alternativen erfordern.
Neben diesen inhaltlichen Aspekten werden auch adressatenbezogene Faktoren genannt. So wird berichtet, dass UML als weniger geeignet angesehen wird, wenn sich Darstellungen an das Management oder die Geschäftsführung richten. In diesem Zusammenhang wird auch auf fehlendes IT-bezogenes Vorwissen auf Managementebene verwiesen, welche das Verständnis erschweren kann. Zudem wird darauf hingewiesen, dass der Einsatz von Modellierung ineffizient sein kann, wenn Kommunikationspartner die verwendete Modellierungssprache nicht beherrschen.

Ein weiterer Themenbereich betrifft die Verständlichkeit sowie den Schulungs- und Lernaufwand. Nach Aussagen der Befragten ist ohne entsprechende Schulung häufig keine korrekte Interpretation von Modellen möglich. Gleichzeitig wird angemerkt, dass der Schulungsaufwand bei komplexen Beschreibungssprachen teilweise als nicht angemessen im Verhältnis zum Nutzen wahrgenommen wird. Für geschulte Anwender kann nach Einschätzung einzelner Befragter in bestimmten Situationen auch eine textuelle Beschreibung effizienter sein als eine Modellierung.

Darüber hinaus werden Aspekte der Kommunikation und praktischen Nutzung angesprochen. So wird berichtet, dass Modelle in vielen Fällen durch zusätzliche textuelle Erläuterungen ergänzt werden müssen. Außerdem wird darauf hingewiesen, dass Modellierungssprachen die direkte Kommunikation zwischen Beteiligten nicht ersetzen. Verständnisprobleme lassen sich nach Einschätzung der Befragten nicht allein durch den Einsatz von Modellierungssprachen lösen. Schließlich wird auch eine teilweise geringe Praxisakzeptanz bestimmter Frameworks beschrieben, selbst wenn diese methodisch umfassend ausgearbeitet sind (z. B. das Zachman Framework). \footnote{vgl. Anhang C1 Limitationen von Modellierungssprachen}

\subsection{Herausforderungen}

Wie man bereits in der Tabelle \ref{tab:architektur} erkennen kann, sind viele Äußerungen der Befragten mit Herausforderungen im Zusammenhang mit ihrer Tätigkeit und Modellierungssprachen getätigt worden.
Die folgende Tabelle \ref{tab:C2kategorien} gibt einen Überblick über die Antworten und Anzahl der Nennungen im Bezug auf die unterschiedlichen Herausforderungen.

 \begin{table}[h!]
\centering
\caption{Übersicht der Kategorien nach Anzahl der Nennungen}
\begin{tabular}{c l c}
\toprule
\textbf{Rang} & \textbf{Kategorie} & \textbf{Anzahl der Nennungen} \\
\midrule
1 & Stakeholder \& Kommunikation & 11 \\
2 & Modellpflege \& Synchronisation & 9 \\
3 & Tool \& Integration & 8 \\
4 & Komplexität \& Lernaufwand & 7 \\
5 & Modellqualität (Eindeutigkeit, Redundanz etc.) & 6 \\
6 & Methodische / fachliche Grenzen & 5 \\
\bottomrule
\end{tabular}

\label{tab:C2kategorien}
\end{table}

Die Auswertung der genannten Herausforderungen zeigt, dass Probleme im Zusammenhang mit Stakeholdern und Kommunikation am häufigsten genannt werden. Insgesamt elf Nennungen beziehen sich auf adressatengerechte Darstellung, fehlende Sprachkenntnisse bei Anwendern oder Kunden, mangelndes IT-Know-how auf Managementebene sowie die eingeschränkte Nutzbarkeit von Modellen, wenn die verwendete Modellierungssprache von den Adressaten nicht verstanden wird. Darüber hinaus wird betont, dass Modellierung allein bestehende Verständnisprobleme nicht lösen kann und zusätzliche Abstimmung sowie Feedback erforderlich sind.
An zweiter Stelle stehen Herausforderungen im Bereich der Modellpflege und Synchronisation mit neun Nennungen. Hierzu zählen insbesondere veraltete Modelle, fehlende Aktualisierung, Schwierigkeiten bei der Versionierung sowie Probleme bei der Konsistenz zwischen verschiedenen Modellrepräsentationen oder zwischen Modell und Implementierung.

Ebenfalls häufig, Nennung von acht befragten Softwarearchitekten, genannt werden Aspekte der Toolunterstützung und Integration. Dazu gehören die Komplexität der eingesetzten Werkzeuge, fehlende Integrierbarkeit in bestehende Organisationsstrukturen, Probleme bei der Nutzung mehrerer Tools sowie Einschränkungen durch proprietäre Lösungen oder grafikorientierte Darstellungen, die Wartung und Versionierung erschweren.

Im Bereich Komplexität und Lernaufwand, so 7 der Befragten, werden insbesondere die hohe Komplexität von Modellierungssprachen, die Vielzahl unterschiedlicher Notationen sowie der damit verbundene Schulungsaufwand und die steile Lernkurve als Herausforderungen beschrieben. Auch die korrekte Anwendung der jeweiligen Sprache wird als anspruchsvoll wahrgenommen.
Weitere sechs Nennungen von Befragten betreffen die Modellqualität. Hierzu zählen Anforderungen an Eindeutigkeit, Redundanzvermeidung, präzise formale Darstellung sowie die Einhaltung einer geeigneten Modellierungsreihenfolge. Gleichzeitig wird darauf hingewiesen, dass Modelle in der Praxis teilweise aus Vereinfachungsgründen reduziert dargestellt werden.

Am seltensten werden methodische und fachliche Grenzen von Modellierungssprachen genannt. Hierauf verweisen fünf der Befragten. Dazu gehören Schwierigkeiten bei der Abbildung bestimmter Sachverhalte, etwa dynamischer Abhängigkeiten in Geschäftsprozessen, Parallelität oder spezifischer Datenbankzugriffe, sowie Einschränkungen einzelner Notationen in bestimmten Anwendungsbereichen.

Der folgende Abschnitt zeigt nach den hier beschriebenen Herausforderungen die Verbesserungsvorschläge der Architekten.

\subsection{allgemeine Verbesserungsvorschläge}

Die Befragten haben viele unterschiedliche Aspekte in den Interviews genannt, die sich unter diese Kategorie zusammenfassen lassen. Folgende Punkte spiegeln die Antworten der Architekten wider. \footnote{Anhang C3 Allgemeine Verbesserungsvorschläge}

Die genannten Verbesserungsvorschläge beziehen sich auf unterschiedliche Weiterentwicklungs- und Einsatzaspekte von Modellierungssprachen sowie auf Rahmenbedingungen ihrer praktischen Nutzung. Ein Themenbereich betrifft die Automatisierung und KI-Unterstützung. Hierzu zählen Vorschläge für automatische Redundanzprüfungen, insbesondere im Kontext der BPMN-Modellierung, sowie der Einsatz von KI-Assistenz zur Unterstützung bei der Erstellung und Analyse von Modellen. Genannt werden unter anderem Funktionen zur Aufdeckung verborgener Abhängigkeiten, eine an Codeassistenz angelehnte Unterstützung für Modellierungssprachen sowie sprachgesteuerte Werkzeuge, die zu einem Zeit- und Komfortgewinn beitragen können.

Ein weiterer Schwerpunkt liegt auf der Unterstützung unterschiedlicher Sichten und der Stakeholderorientierung. Vorgeschlagen werden generierbare Views für verschiedene Zielgruppen, beispielsweise für Management oder Betrieb. Ergänzend wird eine vereinfachte Modellierungs- bzw. Präsentationsebene zur Kommunikation mit nicht-technischen Stakeholdern genannt. Zudem wird die Erweiterung von Modellierungssprachen um einen expliziten Verwaltungskontext angesprochen.

Im Bereich der Flexibilität und strukturellen Ausgestaltung der Modellierungssprache werden Vorschläge zur Bereitstellung einer generischen, beispielsweise XML-ähnlichen Basisstruktur mit Erweiterungsmöglichkeiten für spezifische Anwendungsfälle genannt. Darüber hinaus wird angeregt, dass Modellierungssprachen keinen festen Detaillierungsgrad erzwingen, sondern Architektinnen und Architekten die notwendige Detailtiefe situationsabhängig festlegen können. Weitere Punkte betreffen die Vermeidung übermäßig komplexer Metamodelle sowie die Unterstützung rekursiver beziehungsweise fraktaler Modellierungsansätze.

Ein weiterer Aspekt betrifft die Verständlichkeit und einen niedrigschwelligen Einstieg. In diesem Zusammenhang werden UML-konforme Skizzenmöglichkeiten genannt, die dazu dienen können, Verständnisprobleme bei ungeschulten Anwendern zu überbrücken.
Neben den Verbesserungsvorschlägen wird aber von der Hälfte der Befragten geantwortet, dass es keine Verbesserungsvorschläge bei grundlegenden Modellierungsansätzen gibt. Ebenso wird kein Verbesserungsvorschlag an den Modellierungssprachen selbst gesehen.  Es wurde auch erwähnt, dass die Modellierungssprachen bereits sehr umfangreich sind.

\subsection{Verbesserungen Semantik und graphische Notation}

Nur vereinzelt erwähnen die Befragten die Semantik oder die Notation von Modellierungssprachen. Es gibt keine Verbesserungsvorschläge der Softwarearchitekten in diesem Bereich. Die wenigen Fundstellen in den Interviews beziehen sich auf eine Verneinung von Anmerkungen bzw. Verbesserungsvorschlägen zur Semantik. Gleiches gilt für die Antworten mit Hinblick auf Verbesserung der graphischen Notation.

Für Verbesserungsvorschläge bei der graphischen Notation ist folgender Interviewabschnitt aus dem vierten Interview als Ankerbeispiel bezeichnend.

\begin{quote}
\textit{Interviewer: „Hast du dir da schon mal Gedanken gemacht, wie du die, ich sag mal, die grafische Oberfläche, also die grafische Darstellung verbessern könntest? Oder wo du sagst, wenn das jetzt so und so wäre, würde ich es besser verstehen oder würde es ein Dritter besser verstehen?“}

\textit{Softwarearchitekt 4: „In die Richtung nicht. Meistens findet man dann irgendwo so einen Zwischenweg, der nichts mehr mit Standardisierung zu tun hat, um dann eben zielgruppengerecht irgendwo eine Darstellung hinzukriegen, die dann nichts mehr mit Standard zu tun hat.“}

\hfill \textit{(Interview 4, S.5f)}
\end{quote}

\subsection{Verbesserungsvorschläge Werkzeugunterstützung}

Die Kategorie C6 Toolunterstützung ist mit 66 Fundstellen die am häufigsten Auftretende Kategorie. Die Interviewteilnehmer haben eine Vielzahl von Vorschlägen zur Verbesserung der Werkzeuge genannt.  Die Vorschläge wurden in der folgenden Tabelle noch folgende Segmente eingeordnet.

\begin{table}[h!]
\centering
\caption{ Verbesserungsvorschläge nach Anzahl der Nennungen}
\begin{tabular}{c l c}
\toprule
\textbf{Rang} & \textbf{Kategorie} & \textbf{Anzahl der Nennungen} \\
\midrule
1 & Usability, Einfachheit \& Accessibility & 14 \\
2 & Interoperabilität, Standardisierung \& Toolunabhängigkeit & 12 \\
3 & Versionierung, Synchronisation \& Kollaboration & 10 \\
4 & Automatisierung \& KI-Unterstützung & 10 \\
5 & Erweiterbarkeit & 9 \\
6 & Stakeholderorientierung \& Sichten & 7 \\
7 & Integration in Entwicklungsumgebung & 6 \\
8 & Konsistenzprüfung, Governance \& Modellqualität & 5 \\


\bottomrule
\end{tabular}

\label{tab:C6Verbesserungen}
\end{table}

Die genannten Verbesserungsvorschläge für Werkzeuge zur Nutzung von Modellierungssprachen lassen sich in mehrere Themenbereiche einordnen. Am häufigsten werden Anforderungen im Bereich Usability, Einfachheit und Accessibility genannt. Hierzu zählen eine intuitive Bedienung, eine geringe Komplexität, eine niedrigschwellige Nutzung ähnlich einem Whiteboard sowie die Möglichkeit, schnell Skizzen zu erstellen. Zudem werden Aspekte wie Barrierefreiheit, einfache Weitergabe von Diagrammen sowie eine insgesamt benutzerfreundliche Gestaltung der Werkzeuge angesprochen.

Ein weiterer zentraler Themenbereich betrifft die Interoperabilität, Standardisierung und Toolunabhängigkeit. Genannt werden Anforderungen an standardisierte Austauschformate, eine bessere Austauschbarkeit von Tools ohne Anpassung der Modelle sowie die Vermeidung proprietärer Erweiterungen. Darüber hinaus werden offene Schnittstellen, Import- und Exportmöglichkeiten sowie eine generelle Integrationsfähigkeit zwischen verschiedenen Werkzeugen als relevant beschrieben.

Ebenfalls häufig genannt werden Anforderungen an Versionierung, Synchronisation und kollaborative Nutzung. Dazu zählen die Versionierbarkeit von Modellen, die Synchronisation zwischen verschiedenen Tools oder Modellständen sowie die parallele Bearbeitung durch mehrere Nutzer. In diesem Zusammenhang werden auch Rechte- und Zugriffskonzepte sowie Möglichkeiten zur Segmentierung und Sichtbarkeitssteuerung genannt.

Ein weiterer Schwerpunkt liegt auf Automatisierung und KI-Unterstützung. Genannt werden unter anderem KI-Assistenten zur Produktivitätssteigerung, automatische Modellgenerierung, Reverse Engineering aus bestehendem Code, textuelle Definitionen mit automatischer grafischer Darstellung sowie allgemeine Automatisierungsfunktionen zur Zeitersparnis.

Der nächste Themenbereich betreffen die Erweiterbarkeit der, etwa durch Plugin-Schnittstellen, erweiterbare Workflow-Engines oder die Unterstützung zusätzlicher Prozessmodelle.
Schließlich werden auch einzelne Anforderungen an erweiterte Modellierungsfunktionen genannt, darunter die Darstellung großer Modelle, die Abbildung von Zeitverläufen oder die visuelle Unterstützung zur Analyse von Datenflüssen.


Darüber hinaus beziehen sich mehrere Vorschläge auf die Stakeholderorientierung und unterschiedliche Sichten. Hierzu zählen Management- und Facharchitektursichten, automatisch generierbare zielgruppenspezifische Darstellungen sowie ein anpassbarer Detaillierungsgrad abhängig vom jeweiligen Adressaten.

Ein weiterer Themenbereich betrifft die Integration in bestehende Entwicklungsumgebungen. Genannt werden beispielsweise die Verbindung von Modellen mit Code und Dokumentation, die Möglichkeit zur Diagrammerstellung direkt aus Entwicklungsumgebungen sowie die Unterstützung von Funktionen wie SQL-Skript-Einbettung.

Zudem werden Anforderungen an Konsistenzprüfung und Governance formuliert. Hierzu zählen automatische Konsistenzprüfungen, die Einhaltung definierter Modellierungsrichtlinien sowie Funktionen zur Sicherstellung der formalen Korrektheit und zur Verbesserung der Modellqualität.

Aufbauend auf den dargestellten Ergebnissen erfolgt im nächsten Kapitel eine Interpretation und kritische Einordnung der Befunde.
