% !TEX root = main.tex
% !TEX TS-program = pdflatex
% !BIB program = biber


\lstset{language=C}

\chapter{Ergebnisse gemäß der Methodik Vorgehensmodells}


%
%
%Hier wird die Hauptarbeit entstehen
%
%
%Noch einmal prüfen ob der Interviewleitfaden hier passt ?

\section{Interviewleitfaden}

diese Sektion bzw Unterkapitel wurde am 25.01 in das Kapitel 3 verschoben


 

\section{Aussuchen der Zielgruppe/der Interviewpartner}
Um Softwarearchitekten für die Interviews zu finden, wird folgendermaßen vorgegangen. Der Interviewer hat durch seine Tätigkeit im Rechenzentrum der DRV-Bund Zugang zum Architekturboard des Rechenzentrums in Würzburg und zu einem weiteren Architektur-Board einer IT-Abteilung in Berlin. Von den dort tätigen Softwarearchitekten werden zunächst potentielle Interviewkandidaten angefragt. Um verschiedene Blickwinkel auf die Forschungsfragen zu erhalten, werden die Interviewpartner aus verschiedenen Hierarchiebenen und Tätigkeitsfeldern in der Softwarearchitektur gewählt. Darüber hinaus werden sowohl interne Softwarearchitekten als auch externe Mitarbeiter in den Architekturboards zum Interview eingeladen. Mit einem Skype-Anruf wird erfolgt die erste Kontaktaufnahme. In dieser werden die Forschungsfragen vorgestellt und um ein Interview gebeten. Bei Zustimmung wird die Intervieweinladung per Email verschickt.
Zehn Softwarearchitekten konnten auf diesem Weg für ein Interview gewonnen werden, hiervon musste einem der Fragebogen vorab zugeschickt werden, sonst wäre es nicht zu einer Teilnahme gekommen.  


Ziel ist es, so viele Interviews durchzuführen, bis eine Sättigung an neuen Informationen erreicht ist. Ob dieses Ziel in der Forschungspraxis erreicht werden kann, ist allerdings zu hinterfragen, ob dieses Ziel  Hinblick auf den zeitlichen Rahmen der Qualifikationsarbeit erfüllt werden kann. Allerdings liegt der Fokus zunächst auf der Erhebung der Anforderungen von Softwarearchitekten. Da es sich bei den Antworten der Interviewpartner nicht um einen Anspruch auf Generalisierbarkeit oder auf Übertragbarkeit auf die zugrunde liegende Grundgesamtheit handeln soll, sondern um Personen die mit ihrem spezifischen Wissen für die Beantwortung der Forschungsfragen besonders relevant sind, ist eine zweckorientierte Stichprobe fachlich sinnvoll. Geeignete Auswahlkriterien für die Interviewpartner sind Position im Unternehmen, Berufserfahrung und Fachwissen, dies wurde druch die oben genannten Kriterien erfüllt.

\section{Inhaltsanalyse}
