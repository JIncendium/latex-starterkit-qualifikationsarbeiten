% !TEX root = main.tex
% !TEX TS-program = pdflatex
% !BIB program = biber


\lstset{language=C}

\chapter{Anhang Interviewtranskripte}


\section{Interview 1}
\textbf{Termin des Interviews 13.11.2025}

Interviewteilnehmer: Jonathan Brand(Interviewer), Softwarearchitekt 1(SA1)
\\
\\
 \textbf{Interviewer:} Du bist damit einverstanden, dass ich dich aufnehme für die Auswertung?
 \\  \noindent\rule{\textwidth}{0.4pt} \\  \textbf{SA1:} Ja
 \\  \noindent\rule{\textwidth}{0.4pt}\\  \textbf{Interviewer:} Dann danke dir, dass du dir die Zeit nimmst das Interview habe ich in vier Teile geteilt erst mal kommt kurz die Einführung also Ziel und Zweck ist wie es auf dem Titel der Einladung stand die Softwarearchitekten zu interviewen zu Modellierungssprachen zu den Anforderungen an modellierungssprachen und zu den Herausforderungen und verbesserungspotenzialen und du bist heute der erste der ich interviewen darf überhaupt das freut mich besonders und das erste also der der Einstieg ist eigentlich die kurze frage ja wie ist dein beruflicher Werdegang wie ist deine Berufserfahrung im Bereich Softwarearchitektur wie lange bist du da schon drin welche Branche und welche Projekte also mal kurz zu deiner Person 
 \\  \noindent\rule{\textwidth}{0.4pt} \\  \textbf{SA1:} Ok im Hintergrund, ich habe Mathematik und Physik studiert ich habe unter anderem Dr. Luft- und Raumfahrtechnik und der Astrophysik, bei der DRV-Bund bin ich eine ganze Weile in die Rolle des Architekten bin ich zuvor reingekommen und zwar habe ich als Dezernatsleiter. Das Architekturmanagement bei der DRV Bund aufgebaut und war sozusagen der Initiator und der Leiter des ganzen Bereiches. Zudem gehören halt die Rolle Unternehmensarchitektur, Produktarchitekt und Lösungsarchitekt usw. Softwarearchitekt ist hat nur eine Untergruppe der Lösungsarchitekten in dem Sinne und ich bin dem Sinne auch kein eigentlicher Softwarearchitekt oder Lösungsarchitekt sondern ich bin eher so das was man als Unternehmensarchitekt bezeichnet, zu dem die Rolle mitgehört im Rahmen des Projekts auf mein ganzen Lebenslauf und sowas bei euch gehe ich nicht ein, es ist so dass jetzt im Rahmen des Projekts er für RVevolution ich den Kompetenzbereich Architektur übernommen habe und Leiter und in dieser Rolle ich auch dafür verantwortlich war ja entsprechend die Architekten zu rekrutieren einzusetzen wohl intern als auch externe und da sind dann auch sozusagen die entsprechenden Softwarearchitekten mit dabei, dass innerhalb des Projektes gibt es die Rolle des Solutionsarchitekten, des Systemarchitekten dann das dem Produktarchitekten, Softwarearchitekten das sind so die Ebenen des Solutionsarchitekt ist das was sozusagen für so einen kompletten Release-train mit allem drum und dran passiert dann haben wir ja noch diese Arts also die Agile Release Trains. Da sind die Systemarchitekten unterwegs sind die beiden Teams sind dann entweder Produktarchitekten oder Softwarearchitekten im Einsatz und zwar ist die Unterscheidung so, dass bei den Produktarchitekten, das Architekten die ein bisschen mit entwickeln. Die Softwarearchitekten sind bei uns Entwickler, die die ein bisschen Architektur mitmachen so als grob als Aufteilung als Hintergrund 
 \\  \noindent\rule{\textwidth}{0.4pt}\\  \textbf{Interviewer:} Dann gehe ich mal in die nächste Frage, also du hast gemeint deine Rolle ist etwas übergeordnet über den Softwarearchitekten kannst du die also deine Rolle in Bezug zu der Softwarearchitektur genauer beschreiben?
 \\  \noindent\rule{\textwidth}{0.4pt} \\  \textbf{SA1:} Ja, als Unternehmensarchitekt habe ich bin ich die klammer so von Facharchitektur bis runter zur IT-Architektur, die sich aufteilt in Informationssystems-Architektur, Technologie-Architektur und Betriebs-Architektur. Der Unternehmensarchitekt mache ich die klammer überall und im Prinzip ist es so, dass Unternehmensarchitekten setzen die Rahmen oder die Leitplanken für die anderen Architekten den zugehenden den Standards und Technologien entsprechend vorgegeben werden wie zum Beispiel, dass man sagt es sind folgende Technologien im Sinne von Skriptsprachen zu verwenden es sind folgende Technologien zu verwenden im Basis auf rationale Datensystem und so weiter.
 \\  \noindent\rule{\textwidth}{0.4pt}\\  \textbf{Interviewer:} Betrifft es auch die die Vorgaben für Modellierungssprachen, was die...
 \\  \noindent\rule{\textwidth}{0.4pt} \\  \textbf{SA1:} Ja
 \\  \noindent\rule{\textwidth}{0.4pt}\\  \textbf{Interviewer:} Dann benutzen dürfen 
 \\  \noindent\rule{\textwidth}{0.4pt} \\  \textbf{SA1:} Genau
 \\  \noindent\rule{\textwidth}{0.4pt}\\  \textbf{Interviewer:} Dann welche Modellierungssprachen verwendest du denn oder also zum einen welche werden im Projekt verwendet oder welche hast du in der Vergangenheit schon verwendet?
 \\  \noindent\rule{\textwidth}{0.4pt} \\  \textbf{SA1:} Also was ich verwendet habe, ist alles Mögliche von aktuell Python über Pearl und so weiter bis runter zu den klassischen Programmiersprachen Fortran C, C++, Java und so weiter. Im Projekt liegt der Fokus momentan auf JavaScript, Java, in dem Bestandssystem ist es noch Cobol als alte hergebrachte Sprache in dem KI Umfeld setzen wir jetzt verstärkt auf Python, so dass im Einsatz kommt und historisch hatten wir gewachsen, dass wir im Systemadministrationsumfeld die Pearl einsetzen. 
 \\  \noindent\rule{\textwidth}{0.4pt}\\  \textbf{Interviewer:} Und bei Modellierungssprachen, also wenn ich jetzt Richtung UML oder SYSML oder BPMN nehme was ist da Mittel der Wahl 
 \\  \noindent\rule{\textwidth}{0.4pt} \\  \textbf{SA1:} Da ist dann wir nutzen Standard UML insbesondere für die Diagramme wie Klassendiagramme, Sequenzdiagramme und so weiter ansonsten ist für uns ganz klar BPMN 2.0 für die Geschäftsprozessmodellierung der Standard, den wir dafür einsetzen ansonsten machen wir aber auch nutzen wir auch mal das UML zum Beispiel aber primär versuchen wir uns auf den Standard UML zurückzuziehen die daraus entsprechend.
 \\  \noindent\rule{\textwidth}{0.4pt}\\  \textbf{Interviewer:} Wie sah da der Entscheidungsprozess aus welche Modellierungssprache da genutzt wird?
 \\  \noindent\rule{\textwidth}{0.4pt} \\  \textbf{SA1:} Der Entscheidungsprozess für Standards und Technologien jetzt im Projekt sah so aus dass ich den initialen Tech-Stack für Standards und Technologien geschrieben habe und es einfach reingeschrieben habe weil Abstimmungsprozesse dauern ewig und dauern lange und so eine Sachen wie bpm 2.0 war etabliert schon auch in der Rentenversicherung als Ganzes schon vorhanden und festgelegt und das wurde jetzt nur für das Projekt sozusagen übernommen die anderen Sprachen wie jetzt Standard UML ist halt einfach jetzt reingekommen weil es ist halt weltweit sozusagen weltweit der Standard für die Beschreibung von halt Systemen und Modellen sei es ist ml oder so weiter da werden einfach reingeschrieben dass uml nutzen dann habe 
 \\  \noindent\rule{\textwidth}{0.4pt}\\  \textbf{Interviewer:} ich noch eine also ich werde auch ein paar allgemeine Fragen stellen, Fragen definitorischer Art wie du es wie du es definierst weil der Fokus auch natürlich darauf ist wie sieht ein Architekt bestimmte Dinge und da ist eine der Fragen zu ab wann beginnt für dich Modellierung
 \\  \noindent\rule{\textwidth}{0.4pt} \\  \textbf{SA1:} Das ist eine gute Frage als Architekt beginnt es dann wenn ich anfange mir das erste Modell zu überlegen zu machen das heißt dass ich ein allererstes Diagramm aufmale wie zum Beispiel ein Klassendiagramm überlegen welche Klassen ich drin habe oder wenn ich von der nutzerreise komme und anfangen erst mal sozusagen mir so ein Sequenzdiagramm zu erstellen oder wenn die Fachseite kommt und sie sagt ich habe ja meinen Fachprozess denn und den abbilden BPMN dampf würde das entsprechend beginnen das hat unterschiedliche Startpunkte nach welcher Richtung man kommt
 \\  \noindent\rule{\textwidth}{0.4pt}\\  \textbf{Interviewer:} welche Eigenschaften sind denn besonders wichtig bei, bei der Modellierungssprache also wenn wir jetzt Richtung Ausdrucksstärke oder Toolunterstützung 
 \\  \noindent\rule{\textwidth}{0.4pt} \\  \textbf{SA1:} Standardisierung und Austauschbarkeit von Tools das ist wichtig denn wenn ich was modelliere in der Standardsprache muss es im Zweifelsfall und ich jetzt erst mit einem Tool einsetzen muss ich im Zweifelsfall in Lage sein das Tool auszutauschen ohne das ganze Modell anpassen zu müssen das heißt wenn ich ein Geschäfts Prozess in BPMN modelliert habe und ich wechsle das Tool zum Beispiel von einer Workflow-Engine zum Camunda oder so weiter hin muss es funktionieren auf dem Standard ohne dass ich den Prozess anpassen und neu modellieren muss 
 \\  \noindent\rule{\textwidth}{0.4pt}\\  \textbf{Interviewer:} und welche was wäre die die Anforderung die du speziell mit Blick auf deinen, deinen Job am meisten an der Modellierungssprache hast was muss, muss, muss eine Modellierungssprache unbedingt mitbringen 
 \\  \noindent\rule{\textwidth}{0.4pt} \\  \textbf{SA1:} ähm sie muss wie gesagt es muss ein Standard sein der muss offen sein der muss offen beschrieben sein der Standard der muss nachvollziehbar sein er darf nichts proprietäres enthalten und es muss relativ einfach und intuitiv sein das heißt ich muss nicht Verklausulierung bis zum geht nicht mehr und sonderlocken haben sondern ich muss es einfach haben ich muss überschaubaren Ansatz ähm ähm ähm ähm ähm rat an Tools haben oder Symbol haben oder Befehle haben nicht nutzen kann es darf nicht zu sehr ausufernd sein
 \\  \noindent\rule{\textwidth}{0.4pt}\\  \textbf{Interviewer:} ähm hast du von deinen äh Kollegen mitbekommen ob es äh je nach Zielgruppe also welcher äh ich sag jetzt mal ob es jetzt ein Solutionsarchitekt ist oder ein Softwarearchitekt ähm gibt es da unterschiedliche Anforderungen also hast du mit deinen Kollegen schon drüber geredet worauf die ähm einen besonderen Fokus legen
 \\  \noindent\rule{\textwidth}{0.4pt} \\  \textbf{SA1:} ähm ja mit nicht mit allen aber mit einigen ähm da ist die Stimmung zweigeteilt und der eine teile kennt halt irgendein Tool und will das möglichst weiter nutzen den Spezifika was nicht unbedingt optimal ist und bei den anderen die halten sich im großen Ganzen an die Standards die die einzelnen sprachen vorgegeben haben und versuchen daraus auch entsprechend Toolunabhängig zu modellieren und darzustellen
 \\  \noindent\rule{\textwidth}{0.4pt}\\  \textbf{Interviewer:} ähm ähm wenn ihr mit uml arbeitet mit welchem Programm arbeitet ihr also gibt es irgendeinen du hast Camunda erwähnt ähm gibt es irgendeine Software die ihr da ähm speziell benutzt
 \\  \noindent\rule{\textwidth}{0.4pt} \\  \textbf{SA1:} also für die Workflow-Engine nutzen wir jetzt Camunda ansonsten nutzen wir innerhalb des Programms momentan visual paradigm s für die Modellierung in uml ist aber nicht so dass das für alle Zeit festgeschrieben ist das können wir können das auf perspektive stellt ein anderes Tool austauschen momentan ist das damals bei einer Beschaffung rausgekommen deswegen setzen wir es ein.
 \\  \noindent\rule{\textwidth}{0.4pt}\\  \textbf{Interviewer:} ähm dann was habe ich noch an, an wichtigen Fragen? ähm wie wichtig sind dir die also das Generalisierung als eine der wichtigsten Punkte genommen dass du im Zweifel halt eine Schnittstelle hast ähm wie wichtig sind dir die anderen formalen Anforderungen also jetzt Korrektheit Redundanzfreiheit Wartbarkeit
 \\  \noindent\rule{\textwidth}{0.4pt} \\  \textbf{SA1:} ähm ist für mich eine essentielle Voraussetzung ich gehe davon aus dass wenn ich etwas habe was ein Standard ist dass er genau diese Sachen erfüllt ist muss Redundanz frei sein es darf nicht zwei Wege zum gleichen ziel geben wenn ich das modellieren will dann gibt es dafür eine Lösung im Standard und die ist zu verwenden auch die Wartbarkeit ähm das ist so ein Thema ähm wenn ich Modelle habe muss ich sie regelmäßig anpassen können aber das ist nicht so dass es mit alles selbsterklärend sein wird im Zweifelsfall muss da entsprechend Dokumentation der entsprechenden Ersteller gearbeitet werden
 \\  \noindent\rule{\textwidth}{0.4pt}\\  \textbf{Interviewer:} ähm das heißt ähm dass so die anwenderbezogene Anforderung ist dass die dass der das für den Endnutzer verständlich sein sollte also intuitiv ist ja ist ja noch mal was anderes aber
 \\  \noindent\rule{\textwidth}{0.4pt} \\  \textbf{SA1:} ja ist die Frage der Endnutzer ist äh wenn die Architekten Modelle machen sind sie im Prinzip ja primär in Richtung der Entwickler die so umsetzen sollen oder bereitstellen sollen für die muss das verständlich sein wenn man sagt die Endnutzer sind Zweifelsfall die Anwender der Fachanwender draußen da muss es das nicht sein der muss im Zweifelsfall nur wenn er die jetzt nich in den Business-Process-Language kann und er sich dann den Prozess anguckt da muss das kann ja nichts verstehen aber ich es ist nicht so dass die Modelle die erstellt werden jetzt von jedem Fach-User verstanden werden müssen draußen sind es hängt wie gesagt das ziel hängt davon ab für wen das Modell gemacht ist und wo es für es eingesetzt wird
 \\  \noindent\rule{\textwidth}{0.4pt}\\  \textbf{Interviewer:} ok, also ich spreche jetzt nicht vom normalen Endanwender also nicht vom Fachbereich sondern erstmal in eure Richtung für den Architekten also wie gut ist die Verständlichkeit von dem was sie nutzt für den Architekten und dann im zweiten Schritt genau den Entwickler
SA:1aber die muss gegeben sein für den Architekten
 \\  \noindent\rule{\textwidth}{0.4pt}\\  \textbf{Interviewer:} wenn du an die bestehende Modellierungssprache denkst wo ist aus deiner Sicht also wenn du damit arbeitest das größte das größte Problem oder wo sind die Herausforderungen wenn du mit uml zum Beispiel arbeitest
 \\  \noindent\rule{\textwidth}{0.4pt} \\  \textbf{SA1:} das Problem ist das abbilden von den Gedanken die man hat und von den Sachen die man im Kopf hat in die Modellierungssprache das ist meist eine größte Herausforderung weil man sich denn man hat was im Kopf das hat man meist einen planen und Ideen man muss aber dann wenn man sie in der uml Sprache oder generell Sprache abbilden muss man sich viele Sachen klar machen und festlegen und konkretisieren was bei vielen erstmal im Kopf nicht der Fall ist das ist bei mir so meistens die größte Herausforderung dass das formal zusammengeführt werden muss oder ob man sich Gedanken machen muss um die Sachen Redundanzfreiheit oder an der freiheitsfrage reduzieren und so weiter dass man darum kümmern muss
 \\  \noindent\rule{\textwidth}{0.4pt}\\  \textbf{Interviewer:} Würde dir dafür ein Beispiel, ein konkretes einfallen?
 \\  \noindent\rule{\textwidth}{0.4pt} \\  \textbf{SA1:} Ja wenn ich zum Beispiel ein Klassendiagramm mache und ich dann grob bei den Geschäftsprozessen klaren bin und dann aber von Abhängigkeiten Arbeit aber dann gewisse Abhängigkeiten annehme aber mir nicht alle bewusst sind und ich mir wenn ich das hier mal angucke ich dann erstmal lustig aber dann entsprechende Lücken drin auftauchen würde ich das so gut klar zu machen und die Lücken dann sinnvoll auszufüllen ist meistens das was die große Herausforderung ist
\\ \textbf{Interviewer:} ich frage nochmal kannst du es nochmal konkretisieren also du kannst auch gerne ins Detail gehen
 \\  \noindent\rule{\textwidth}{0.4pt} \\  \textbf{SA1:} fällt mir spontan weil ich in letzter Zeit nicht viel gemacht habe gar nichts ein also es sind eher so allgemeine Sachen die mir da im Kopf umherschwirren oder das was andere an mir erzählen dass das Probleme an Sachen ihre Gedanken fest abzubringen kriegs zu formulieren ich glaube daran ist das große Problem
 \\  \noindent\rule{\textwidth}{0.4pt}\\  \textbf{Interviewer:}: hast du noch also wie geht ihr mit oder wie gehst du mit den ich sag mal das mit den Unklarheiten generell um tauscht dich dann mit den anderen aus oder ähm gibt es einen ich sag mal ein Musterweg wie man wie man denn die richtige Lösung zum Papier bringt also irgendein Workflow oder so?
 \\  \noindent\rule{\textwidth}{0.4pt} \\  \textbf{SA1:} ähm Muster weg gibt es nicht ähm es ist problemabhängig oft äh tauscht man sich mit anderen Architekten denn aus die ähnliche Probleme haben oder von denen man weiß dass sie mal vor ihnen zu leben standen um zu gucken wie sie es gelöst haben ähm mitunter manchmal ist aber auch selber dass man so versucht sich selber durchzubeißen aber das meiste ist halt der Austausch
 \\  \noindent\rule{\textwidth}{0.4pt}\\  \textbf{Interviewer:} ähm neben dem es auf den aufs Papier bringen beziehungsweise also auch vom Kopf in die Umsetzung bekommen gibt es bestimmte Sachen die ihr als Architekten immer berücksichtigt also ähm ich habe ein Interview gehört zu wie man wie man Systeme ähm robuster macht ja habt ihr da ein Framework in diese Richtung?
 \\  \noindent\rule{\textwidth}{0.4pt} \\  \textbf{SA1:} bisher nicht würden gerne sowas aufbauen haben wir bisher nicht weil dazu fehlt uns momentan die Zeit und Ressourcen erst entsprechend aufzubauen wir würden das so dass wir momentan das wir haben, haben wir gruppe da und das gucken wir auch dass wir nach nutzen aber wir haben kein dediziertes Framework bisher 
 \\  \noindent\rule{\textwidth}{0.4pt}\\  \textbf{Interviewer:} Ähm lässt sich sowas durch eine Modellierungssprache abbilden?
 \\  \noindent\rule{\textwidth}{0.4pt} \\  \textbf{SA1:} ähm einige Sachen ja also wenn man sich die Architekturprinzipien Richtlinien entsprechend angucken vorgibt kann man Sachen abbilden das hängt aber dann wieder vom Diagramm ab welches man nutzt also ich glaube dass man im Klassendiagramm viele Sachen abbilden kann ähm SYSML auch aber wenn ich jetzt mich in so ein Sequenzdiagramm habe was er sozusagen den Informationsfrist darstellt kann es in Bezug auf diese nicht funktionalen Anforderungen etwas problematischer werden.
 \\  \noindent\rule{\textwidth}{0.4pt}\\  \textbf{Interviewer:} Ähm könntest du da würde dir da ein Beispiel einfallen wenn es problematischer wird es immer interessant wenn es um Fehler geht 
 \\  \noindent\rule{\textwidth}{0.4pt} \\  \textbf{SA1:} wenn ich zum Beispiel mir angucke dass ich das habe wo ich jemand sozusagen sich zertifizieren muss authentifizieren muss dann und ich jetzt von der Systemseite her aber Anfang habe ich muss Sachen redundant ausliegen und so weiter kann ich das im SYSML also Systemmodellierung im Systemmodell relativ gut darstellen auch im Klassenmodell noch einigermaßen im Sequenzdiagramm ist das so ein Redundanz schwerer darzustellen da habe ich ja den Sprung von einzelnen bahn zu dem anderen hin ähm und äh dann Redundanz hinter zu hinterlegen schwer ich denke es muss gucken was man genau mit dem jeweiligen Diagramm darstellen will gerade im Umfeld da gibt es ja dynamische und statische Diagramme und da muss man genau gucken was ist das Ziel was ich mit dem Diagramm erreichen und dokumentieren will
 \\  \noindent\rule{\textwidth}{0.4pt}\\  \textbf{Interviewer:} ähm gibt es eine Möglichkeit oder wenn du wenn du an Verbesserungspotenzial denken würdest ähm wie wichtig wäre ähm quasi so ein Redundanzfilter in der Modellierungssprache dass der dir ja vielleicht automatisch sagt hey da ist eine Schleife drin wenn das möglich wäre also ein bisschen wünsch dir was 
 \\  \noindent\rule{\textwidth}{0.4pt} \\  \textbf{SA1:} ähm also so ein automatischer Redundanzfilter, der Redundanz und was ausfiltert wäre gut oder sogar sehr gut das wäre eine Sache die wirklich ähm viel bringen würde ähm ist aber würde ich auch insbesondere Fokus sehen bei dem BPMN-Modellierung weil durch die Businessprozesse weil da erstmal ist es so dass da doch sehr oft auch in den Inhalten Redundanzen sind oder doppelschleifen oder unnötige Durchläufe sind dafür wäre das genau das richtige Tool sowas zu haben ähm hättest du ähm in Bezug auf Semantik von äh in der Modellierungssprache wie uml 
 \\  \noindent\rule{\textwidth}{0.4pt}\\  \textbf{Interviewer:} äh hast du da Anmerkungen oder äh Ideen was also was gefällt dir gut was, was äh gefällt dir nicht so gut?
 \\  \noindent\rule{\textwidth}{0.4pt} \\  \textbf{SA1:} Gut gefällt mir, dass sozusagen das alles im Prinzip so XML artige Strukturen sind weil wenn man einmal verstanden hat wie XML funktioniert kann man sozusagen das im Prinzip übertragen auf alle anderen das finde ich gut ähm manchmal erzeugen so eine Sachen relativ großen Overhead das heißt wenn ich eine XML Struktur abbilde mit den vielen Unterstrukturen kann es manchmal sehr viel werden wenn es manchmal etwas kleines ist da wäre so ein Mittelweg manchmal schöner 
 \\  \noindent\rule{\textwidth}{0.4pt}\\  \textbf{Interviewer:} ähm dann habe ich noch eine Frage arbeitest du wenn du mit äh Modellierungssprache arbeitest eher ähm mit der grafischen Oberfläche oder bist du ähm ich sag jetzt mal oldschool im Editor unterwegs
 \\  \noindent\rule{\textwidth}{0.4pt} \\  \textbf{SA1:} ähm früher mehr Editor jetzt mittlerweile mehr grafische Oberfläche 
 \\  \noindent\rule{\textwidth}{0.4pt}\\  \textbf{Interviewer:} hast du ähm warum bist du auf die grafische gewechselt
 \\  \noindent\rule{\textwidth}{0.4pt} \\  \textbf{SA1:} ähm Zeit und äh in unserer Umgebung gibt es die Editor mit denen ich normalerweise arbeite nicht
 \\  \noindent\rule{\textwidth}{0.4pt}\\  \textbf{Interviewer:} Ok ähm das Programm mit dem mit dem du arbeitest ähm Camunda und das andere habe ich äh
 \\  \noindent\rule{\textwidth}{0.4pt} \\  \textbf{SA1:} visual paradigm
 \\  \noindent\rule{\textwidth}{0.4pt}\\  \textbf{Interviewer:} Visual paradigm,was würdest du dir da wünschen was das visual paradigm äh noch mitnimmt noch mitbringt
 \\  \noindent\rule{\textwidth}{0.4pt} \\  \textbf{SA1:} ähm äh wenn das eine Workflow-Engine hätte wie Camunda wäre das ganz gut oder wenn diese oder wenn die Interoperabilität zwischen den beiden Tools wesentlich besser wäre und sie sich mehr auf Standard zurückziehen würden
 \\  \noindent\rule{\textwidth}{0.4pt}\\  \textbf{Interviewer:} also quasi eine Schnittstelle zwischen Camunda und visual paradigm
 \\  \noindent\rule{\textwidth}{0.4pt} \\  \textbf{SA1:} genau um Sachen die jetzt in visual paradigm modelliert äh wurden sind äh nahtlos zu übernehmen zu können ohne nachzuarbeiten momentan ist so man kann zwar die Grundstücke wenn Standard übernehmen muss aber aufgrund der doch teilweise proprietären Ansätze von Camunda und visual paradigm aber da hat so ein bisschen proprietäre Sachen mit drin entsprechend Anpassungen machen deswegen ist es immer schwer zu sagen ich habe jetzt ein führendes System und da hätte auch alles ähm die anderen Tools und anderen Systeme ab das ist entsprechend aufwendig das ist das was wohl der Standard mehr eingehalten werden müsste und eventuell auch übergreifend konzipiert und gedacht werden müsste
 \\  \noindent\rule{\textwidth}{0.4pt}\\  \textbf{Interviewer:} ähm bei der bei der Toolunterstützung ähm wenn du an Usability Performance oder Kompatibilität haben wir jetzt gerade kurz angesprochen aber wie sieht es bei Usability und Performance aus habt ihr mit dem mit den beiden habt ihr da Probleme oder? ähm 
 \\  \noindent\rule{\textwidth}{0.4pt} \\  \textbf{SA1:} ähm Performance ist gut bei weitem also jedenfalls wie jeweils Einsatzfeld ja äh Usability ist ähm etwas gewöhnungsbedürftiger bei beiden weil jeder sozusagen seine Eigenheiten hat seine eigenen Oberflächen etwas mitbringt die nicht immer intuitiv sind das gilt für diese Tool zumindest leider so
 \\  \noindent\rule{\textwidth}{0.4pt}\\  \textbf{Interviewer:} ähm was würdest du dir Richtung Induktivität äh wünschen also was ähm ich meine es sind komplexe, komplexe Programme ja ist immer die Frage wie intuitiv kann so etwas sein?
 \\  \noindent\rule{\textwidth}{0.4pt} \\  \textbf{SA1:} genau das das ist die gute Frage es kann sich nicht hundertprozentig eindeutig sein aber ähm zum Beispiel wenn ich äh ein menüleintrag Format habe dann möchte ich auch alle Formate darunter finden wenn ich gewisse einzelne Formate wieder woanders suchen müssen 
 \\  \noindent\rule{\textwidth}{0.4pt}\\  \textbf{Interviewer:} ähm ich gehe nochmal auf deine Erfahrungen hinsichtlich der reinen ich sag mal textbasierten Sprache zurück was würde dir da einfallen was du ich sag mal was dich was an Verbesserungsmöglichkeiten würden dir einfallen
 \\  \noindent\rule{\textwidth}{0.4pt} \\  \textbf{SA1:} ähm es würde auch eher in Richtung gewisse Toolunterstützung gehen wenn man die Sachen sozusagen wo ich mit Editor bearbeitet ähm verbesserte Unterstützung so wie es ja jetzt bei code-assistant sowas ist da eventuell auch die Unterstützung bei der Modellierung zu haben das wäre wäre so eine Sache das ist noch sehr rudimentär bei vielen Editoren
 \\  \noindent\rule{\textwidth}{0.4pt}\\  \textbf{Interviewer:} okay ähm gibt's anwendungsfälle die dir einfallen wo jetzt sowohl das Tool ähm von Camunda als auch das andere visual paradigm ähm an seine Grenze kommt
 \\  \noindent\rule{\textwidth}{0.4pt} \\  \textbf{SA1:} ja es gibt so eine Grenzfälle und zwar wenn man sich mit ähm Modellen zum Beispiel mit Decision-Modellen oder sowas beschäftigt da haben beide ihre Einschränkungen aber es gibt ja nicht nur die äh es gibt die UML SYSML gibt es ja die ganze Sachen es gibt auch so ein Decision also ein Entscheidungsmodell und da sind beide noch zwar sie können das beide aber sie haben beide Luft nach oben wenn man es optimieren könnte 
 \\  \noindent\rule{\textwidth}{0.4pt}\\  \textbf{Interviewer:} ähm kannst du mir das Entscheidungsmodell äh Entscheidungsmodell kurz erläutern
 \\  \noindent\rule{\textwidth}{0.4pt} \\  \textbf{SA1:} ja es gibt zum Beispiel so ein Decision-Modell wo man modellieren kann wenn jetzt a nach b kommt und das sozusagen auch gleich in sozusagen eine Ausführung bringen will als äh bei Camunda so eine Workflow engine dass diese Modellierung zwar im Papier eigentlich was gut funktioniert aber die Abbildung in diese Workflow engine noch nicht funktioniert und nicht sauber funktioniert das heißt wenn ich jetzt große verzweigungsbäume habe wie ähm wie jetzt bei uns sind wir oft haben dass wir haben sieht kann man sowas oft mit ifs und else ganz vielen abbilden ähm das kann man aber auch in so einer Modellierungssprache bilden und da das zu übersetzen ist nachher eine Workflow engine dass du da haperst du meistens
 \\  \noindent\rule{\textwidth}{0.4pt}\\  \textbf{Interviewer:} ähm würden dir da Verbesserungsmöglichkeiten oder Vorschläge äh einfallen also wie wenn du wenn du es lösen könntest wie du wie würdest du es machen
 \\  \noindent\rule{\textwidth}{0.4pt} \\  \textbf{SA1:} ich würde gucken dass sich die Workflow engines die momentan meistens auf bpnm ausrichtet sind auch wirklich auf so eine andere Modelle ausrichten würde ich würde versuchen das allgemein dazu halten gerade bei den dynamischen Modellen
 \\  \noindent\rule{\textwidth}{0.4pt}\\  \textbf{Interviewer:} wie ähm wie kann ich mir das wie kann ich mir das als nicht Softwarearchitekt vorstellen wie kann man da sowas dynamischer halten
 \\  \noindent\rule{\textwidth}{0.4pt} \\  \textbf{SA1:} ähm das ist nachher die Toolsache das von der Modellierung her ist es okay da kann ich dieses das schöne Modell abreden ähm aber diese Übersetzung nachher in die Workflow engine die ein Tool nachher ausführen soll oder so das funktioniert noch nicht so ohne weiteres oder zumindest ist mir kein Tool bekannt was das bisher kann es gibt markt Tools geben die das fern 
 \\  \noindent\rule{\textwidth}{0.4pt}\\  \textbf{Interviewer:} ähm wie sehr seid ihr schon äh in Kontakt mit ich sag jetzt mal äh KI unterstützten Tools gekommen oder wie sehr ähm hat es das schon betroffen
 \\  \noindent\rule{\textwidth}{0.4pt} \\  \textbf{SA1:} ähm nehmen wir äh wir fangen gerade mit dem Thema an das zu machen das heißt wir versuchen jetzt KI-tools einzubilden generell als äh Codeassistenz also als Codeassistenz und das soll dann möglichst auch auf die ganzen äh entsprechenden Modellierungssprachen sowas auch abgebildet und erweitert wird aber da stehen wir noch hier am Anfang
 \\  \noindent\rule{\textwidth}{0.4pt}\\  \textbf{Interviewer:} ähm hättest du ähm komplett also welche Ideen hast du in Bezug auf Modellierung oder hast du ganz neue Ansätze in Richtung Modellierung ähm ja lass ich mal so stehen ohne dass ich noch was dazu sage ähm
 \\  \noindent\rule{\textwidth}{0.4pt} \\  \textbf{SA1:} neue Ansätze glaube ich nicht weil ich glaube die Ansätze die da sind sind gut und vernünftig man muss sie nur richtig anwenden und das ist meistens das Problem dass Sachen nur halbherzig gemacht werden beziehungsweise nicht wenn man was macht nicht richtig zu Ende denkt sondern einfach vieles auf schnell-schnell macht auch gerade wenn man es mit grafischen Oberflächen modelliert dann sieht es immer sehr schnell sehr schön und gut aus aber manchmal vergisst man Details die meist sehr wichtig sind
 \\  \noindent\rule{\textwidth}{0.4pt}\\  \textbf{Interviewer:} ähm da wären wir wieder bei ich sag jetzt mal dass es sehr schnell gut aussieht das geht ja dann wieder in die ähm Frage nach der Robustheit ja ist es dann wenn man es umsetzt oder wenn es die Entwickler entwickeln sollen nach den äh nach den Vorgaben auch so dass es ich sag jetzt mal den Test der Praxis standhält
 \\  \noindent\rule{\textwidth}{0.4pt} \\  \textbf{SA1:} Genau
 \\  \noindent\rule{\textwidth}{0.4pt}\\  \textbf{Interviewer:} ähm dann neben den Tools die du ähm nutzt oder nutzt beruflicher Natur hast du ähm noch mit anderen gearbeitet?
 \\  \noindent\rule{\textwidth}{0.4pt} \\  \textbf{SA1:} ähm ja früher haben wir ganz Hardcore immer vieles gemacht wirklich im Editor einfach so runter gemacht diese Mark-up-Languages um gerade um Modelle aus was zu erstellen ähm und auch Datenmodelle in XML abzubilden das war früher meistens auf äh Editor Basis ja wir haben zwar mal ein bisschen auch mit anderen gespielt also wir kommen auch mit so anderen äh BPNM-Modellen ähm ähm die sind aber dann sehr oft äh entweder zu schnell in eine reine Proprietär-Richtung abgedriftet ähm zum Beispiel äh wie zum Beispiel äh wie zum Beispiel das Adonis da kann man auch so wunderbar Geschäftsprozesse machen und ist aber nur Proprietär da ist mit Kompatibilität und Wechseln kaum was möglich ähm ähm ansonsten haben sich meistens äh äh außerhalb der Arbeit immer meistens eher wirklich die Rein textbasierten die Editor-Ansätzen sozusagen ja ähm
 \\  \noindent\rule{\textwidth}{0.4pt}\\  \textbf{Interviewer:} ähm gibt es ähm bei deiner Arbeit Literatur die du ähm als Softwarearchitektur als Softwarearchitekt oder als Architekt äh empfehlen ähm empfehlen kannst ähm
 \\  \noindent\rule{\textwidth}{0.4pt} \\  \textbf{SA1:} ähm ja es gibt so eine Standardwerke ähm hier hinten habe ich eins hier UML 2.5 du weißt ob du das erkennen kannst(Anmerkung SA1 deutet auf Regal hinter sich)
 \\  \noindent\rule{\textwidth}{0.4pt}\\  \textbf{Interviewer:} Ja
 \\  \noindent\rule{\textwidth}{0.4pt} \\  \textbf{SA1:} äh das wäre eins das man machen kann äh ansonsten ist es viel ähm im Internet gucken ähm also ich persönlich meine meistens wieder in so einen Cheat Sheets die es denn gibt wenn ich mir die angucke und damit Sachen ausprobiere das ist glaube ich eher das ist das was ich eigentlich meistens natürlich darüber ran zu hangeln an die Sachen
 \\  \noindent\rule{\textwidth}{0.4pt}\\  \textbf{Interviewer:} ähm also gehst du eher auf die Suche nach Best Practices
 \\  \noindent\rule{\textwidth}{0.4pt} \\  \textbf{SA1:} Best Practices ja kann man so sagen und äh auch äh auch offene Standards oder offene Sachen also ich mag keine äh Proprietären Sachen sondern ich bin Freund von Open Source und es sollte alles grundsätzlich im Open Source und sauber dokumentiert sein
 \\  \noindent\rule{\textwidth}{0.4pt}\\  \textbf{Interviewer:} ähm ähm würdest du für dich mal eine Best Practice beschreiben
 \\  \noindent\rule{\textwidth}{0.4pt} \\  \textbf{SA1:} lifestyle selbst und erst ja gerade erwähnt ähm sauber und offen Open Source ähm ähm könnteste da mal ääh reingehen. Ähm ähm ja Ja, wichtig ist, dass alles klar ist, alles offen ist, alles dokumentiert ist und alles eindeutig ist, aber so, dass man es im Zweifelsfall auch nehmen kann und weiterentwickeln kann. Das ist wichtig, dass wenn ich einen rudimentären Satz habe, zum Beispiel in UML, wenn ja auch dann zum Beispiel in so einer Erweiterung in SysML oder sowas drauf kommen, das ist glaube ich das Entscheidende, dass man im Zweifelsfall gucken kann, ich habe die Basis und kann die entsprechend weiter sozusagen nehmen und auch darauf was Neues aufbauen, oder was Weiters aufbauen, die Konzepte weiter zu nutzen. Sowas in der Richtung.
 \\  \noindent\rule{\textwidth}{0.4pt}\\  \textbf{Interviewer:} Okay.
 \\  \noindent\rule{\textwidth}{0.4pt} \\  \textbf{SA1:} Ich könnte mir jetzt auch vorstellen, dass man irgendwann mal, wir haben jetzt auch diese X-Rechte und sowas als das XML-derivate , ich könnte mir durchaus vorstellen, dass man irgendwann mal, was weiß ich, Gesetzestexte oder Rentenversicherungskontexte auch über eine UML oder von UML in der abgeleiteten Modulierungs-Sprache wahrbilden kann.
 \\  \noindent\rule{\textwidth}{0.4pt}\\  \textbf{Interviewer:}: Wir haben da, um da nochmal reinzugehen, wie viele Jahre machst du die Arbeit schon in der Rente?
 \\  \noindent\rule{\textwidth}{0.4pt} \\  \textbf{SA1:} Das mache ich jetzt seit 22 Jahren, aber das Architektur-Management haben wir vor 13 Jahren angefangen aufzubauen, 2012.Da habe ich das Dezernat übernommen und gleich angefangen entsprechend das Architektur-Management aufzubauen.
 \\  \noindent\rule{\textwidth}{0.4pt}\\  \textbf{Interviewer:}: Wie viele Architekten seid ihr insgesamt?
 \\  \noindent\rule{\textwidth}{0.4pt} \\  \textbf{SA1:} Wir waren zwischenzeitlich mal nur für die DRV-Bund knapp so 15 bis 20 Architekten. Momentan gab es eine Unterteilung zwischen Unternehmensarchitekten und Lösungsarchitekten. Bei den Lösungsarchitekten sind wir jetzt, glaube ich, bei 20 bis 25. Unternehmensarchitekten sind wir momentan vier.
 \\  \noindent\rule{\textwidth}{0.4pt}\\  \textbf{Interviewer:} Ich gehe nochmal auf die Herausforderungen ein und die Verbesserungspotenziale. Wenn du dir wirklich eine Sprache wünschen könntest, ein absolutes Wünschen dir was, was muss die können? Was wäre eine Funktion neben der, die du vorhin schon gesagt hast, die sie absolut mitbringen soll? Und jetzt rein von der Sprache und dann natürlich auch ins Tooling?
 \\  \noindent\rule{\textwidth}{0.4pt} \\  \textbf{SA1:} Also definitiv würde ich sagen, sollte auf XML basieren, weil es momentan, glaube ich, das sauberste Austauschformat ist zwischen allen Bereichen. Es sollte so sein, dass es so allgemein ist, dass es für viele Anwendungsfälle funktioniert, sollte aber die Option haben, zum Beispiel als darauf abgelehnt, Systeme auch Spezialisierungen zu haben. Ich könnte mir vorstellen, dass man zum Beispiel was hat für grobe Systeme, dass man ebene da hat und die dann speziell für die einzelnen Sachen sich entsprechend verzweigt, dass man zum Beispiel spezielle Ansätze zum Beispiel haben kann für, was weiß ich, für gewisse Infrastruktursysteme zum Beispiel oder für gewisse Daten. Modellierungen und so weiter, für Datenmodelle, dass man das entsprechend von so einem Standard dann sozusagen kaskadierend ableiten könnte. Ähnlich wie es der UML jetzt auch ist, aber das speziell vielleicht nochmal erweitert auf die Belange von der Verwaltung.
 \\  \noindent\rule{\textwidth}{0.4pt}\\  \textbf{Interviewer:} Okay. Ja, ich habe von meinen Fragen habe ich schon relativ viele gestellt. Ich habe noch ein paar im Peto Erstmal natürlich die Frage nach, welche Softwarearchitekten kannst du mir als weitere Interviewpartner empfehlen? Ich weiß, die Frage hatten wir im Vorfeld schon mal, aber ich hätte gerne nochmal mit.
 \\  \noindent\rule{\textwidth}{0.4pt} \\  \textbf{SA1:} Ja, die gleichen würde ich sagen wieder, also da das angeht. Ansonsten gucken, Thomas Waschau zum Beispiel, der ist jetzt momentan als Systemarchitekt im Projekt unterwegs, also das heißt, er ist auf Artebene, kennt sich aber zum Beispiel gut aus. Aber ansonsten, die Namen, die ich damals genannt habe, sind eigentlich noch Konsequenzen, Sandro Giebel, Niki Klimek und so weiter, Roland Berg, das sind glaube ich die, die entsprechend dabei sind.
 \\  \noindent\rule{\textwidth}{0.4pt}\\  \textbf{Interviewer:} Okay, dann natürlich danke für die Empfehlung. Zwei davon habe ich glaube ich schon im Peto, aber es macht nichts, ist ja gut.Darf ich dir Rückfragen stellen zu dem Interview, wenn bei der Analyse nochmal was aufkommt? Ansonsten, wir sind jetzt relativ schnell durch, ich habe natürlich noch ein paar andere Sachen. Bei Camunda würde mich natürlich interessieren, also ich würde jetzt mal das Interview noch offenlassen, wie seid ihr da draufgekommen? Ich frage, ich habe einer meiner ältesten Freunde arbeitet dort, seit das ein Startup ist schon, also der ist relativ lange dabei.
 \\  \noindent\rule{\textwidth}{0.4pt} \\  \textbf{SA1:} Wir sind drauf gekommen, wir haben geguckt, was gibt es am Markt und was für Tools kommt für uns in Frage. Da haben sich dann glaube ich ein oder zwei herauskristallisiert und da sind wir dann in eine Ausschreibung rangegangen.
 \\  \noindent\rule{\textwidth}{0.4pt}\\  \textbf{Interviewer:} Wie, du wolltest noch was sagen, Entschuldigung
 \\  \noindent\rule{\textwidth}{0.4pt} \\  \textbf{SA1:} Die natürlich dann so eine offene Ausschreibung ist, wir natürlich die Randbedingungen haben, die Vorgaben haben. Dann gibt es Angebote von den jeweiligen Herstellern oder den Leuten, die diese Software vertreiben. Und dann wird geguckt, was da für die Anforderungen am meisten sind. Und dann gibt es immer so einen Faktor zwischen Anforderungen und auch Wirtschaftlichkeit. Das heißt, es gibt Punkte für, wie werden die Anforderungen erfüllt. Dann wird nochmal geguckt, wie sieht es mit der Wirtschaftlichkeit aus. Und da gibt es immer so ein Verhältnis, 70 Prozent für die Fachlichkeit, 30 Prozent Wirtschaftlichkeit. Da gibt es dann so einen Gesamtscore gebildet und darüber die Ausschreibung entschieden. 
 \\  \noindent\rule{\textwidth}{0.4pt}\\  \textbf{Interviewer:} Und jetzt sind wir wieder, jetzt hast du mich auf einen guten Trichter gebracht. Du hast gesagt, da sind zwei rausgekommen am Ende. Das war aber bestimmt nicht die erste Runde.
 \\  \noindent\rule{\textwidth}{0.4pt} \\  \textbf{SA1:} Doch, das war, nein, wir haben vorher natürlich eine entsprechende Recherche gemacht, Produktrecherche, und haben dann entsprechend natürlich von gutem Anfang schon viel rausgehauen. Und aussortiert, wie wir die nicht erfüllt haben.
 \\  \noindent\rule{\textwidth}{0.4pt}\\  \textbf{Interviewer:} Wie viele habt ihr weggewischt? Weißt du das noch?
 \\  \noindent\rule{\textwidth}{0.4pt} \\  \textbf{SA1:} Ich glaube, fünf oder sechs haben wir vorher aussortiert.
 \\  \noindent\rule{\textwidth}{0.4pt}\\  \textbf{Interviewer:} Kannst du mir die nennen?
 \\  \noindent\rule{\textwidth}{0.4pt} \\  \textbf{SA1:} Nee, weiß ich nicht. Das ist schon so lange her. Ich weiß nicht mehr, was der zweite war, der in Frage kam. Aber es war auch so ein offener Anbieter. Und Camunda hat hier relativ schnell das rausgestellt, was für uns am besten passend ist.
 \\  \noindent\rule{\textwidth}{0.4pt}\\  \textbf{Interviewer:} Okay. Kannst du mir da nochmal die wichtigsten Punkte nennen, weswegen es diese Software für die Modellierungssprache geschafft hat?
 \\  \noindent\rule{\textwidth}{0.4pt} \\  \textbf{SA1:} Also den Ausschlag für Camunda hat definitiv gegeben, dass die entsprechende Workflow-Engine dabei war, um die Modelle entsprechend auch gleich ausführen zu können und zu überführen zu können. Das war einer der hauptausschlagenden Punkte. Dann, dass es relativ nah am Standard ist im Gegensatz zu anderen und kaum oder so gut wie keine proprietären Sachen oder Erweiterungen dazu gepackt hat.
 \\  \noindent\rule{\textwidth}{0.4pt}\\  \textbf{Interviewer:} Darf ich mir das so vorstellen, also ich direkt ausführbar wie eine Java-Eclipse, dass ich was reinhaue und dann „Compile“ drücke? Oder wie kann ich mir das vorstellen?
 \\  \noindent\rule{\textwidth}{0.4pt} \\  \textbf{SA1:} Naja, so in die Richtung. Es geht halt so, wie man es mir jetzt im Geschäftsprozess nimmt und denen vorgibt, dass sozusagen die Workflow-Engine diesen Geschäftsprozess wirklich auch nach den Schritten und Aktivitäten abarbeiten kann und auch gucken kann, wo sind eventuell Lücken da drin in dem Prozess, wo fehlen eventuell Informationen und Daten. Das ist sozusagen das Entscheidende. Bis hin zu, dass man nachher so einen Geschäftsprozess über so einen Workflow-Engine auch entsprechend steuern kann, dass sie selbstständig dann weiß, in der Aktivität muss ich jetzt dieses Programm oder diesen User aufrufen und so weiter.
 \\  \noindent\rule{\textwidth}{0.4pt}\\  \textbf{Interviewer:} Welche Kriterien bei der Architektur bedenkt ihr alle? Also wenn ihr sagt, ich meine, ihr macht ja nicht im leeren Raum, sondern ihr wisst, da müssen am Ende zig Millionen Datensätze hin und her gejagt werden.
 \\  \noindent\rule{\textwidth}{0.4pt} \\  \textbf{SA1:} Genau. Wir haben halt für eine Architektur, wir haben einmal die fachlichen Anforderungen und dann die nicht fachlichen Anforderungen, wie zum Beispiel, dass wir wissen müssen, wie die Verfügbarkeit ist, welchen Datendurchsatz muss gewährleistet sein, was haben wir für eine RPO und RTO, das heißt, wenn wir einen Havariefall haben, die lange dort im System ausfallen, bis zu welchem Zeitpunkt müssen Daten wiederhergestellt werden können. Das sind so Sachen, die auf der Betriebsarchitekturebene mitgedacht werden und Anforderungen sind bis hin zu, wo das um die Fachlichkeit funktionieren muss. Und dazwischen haben wir auf allen Architekturebenen, gibt es entsprechende Anforderungen.
 \\  \noindent\rule{\textwidth}{0.4pt}\\  \textbf{Interviewer:} Ist die Betriebsarchitektur auch mit den zwei Tools abbildbar oder braucht das was Separates? 
 \\  \noindent\rule{\textwidth}{0.4pt} \\  \textbf{SA1:} Das braucht eher was Separates.
 \\  \noindent\rule{\textwidth}{0.4pt}\\  \textbf{Interviewer:} Hast du damit auch zu tun?
 \\  \noindent\rule{\textwidth}{0.4pt} \\  \textbf{SA1:} Früher ja, jetzt im Projekt nicht mehr. Weil für die Betriebsarchitektur ist fast ausschließlich das gemeinsame Rechenzentrumsverantwortlichen zuständig und wir als Projekt steuern über den Plattformart die entsprechenden Anforderungen in die Richtung.
 \\  \noindent\rule{\textwidth}{0.4pt}\\  \textbf{Interviewer:} Das heißt, ihr habt also eine klare Trennung zwischen der Softwarearchitektur und der Modellierung in dieser Richtung und der Betriebsarchitektur und der Modellierung.
 \\  \noindent\rule{\textwidth}{0.4pt} \\  \textbf{SA1:} Ja
 \\  \noindent\rule{\textwidth}{0.4pt}\\  \textbf{Interviewer:} Okay, das ist gut zu wissen. Ich habe ja natürlich auch für das gemeinsame Rechenzentrum ein paar mit mir rausgepickt zum Interviewen. Ja, dann bedanke ich mich an der Stelle, dass du dir die Zeit genommen hast. 
 \\  \noindent\rule{\textwidth}{0.4pt} \\  \textbf{SA1:} Und wenn du noch mal Fragen kommst, einfach melden.
 \\  \noindent\rule{\textwidth}{0.4pt}\\  \textbf{Interviewer:} Ja, vielleicht zu den Architekturstandards. Die habt ihr im Intranet öffentlich hinterlegt?
 \\  \noindent\rule{\textwidth}{0.4pt} \\  \textbf{SA1:} Ja, die sind im DIMA(Anmerkung steht für digitaler Marktplatz, das wiki der DRV) hinterlegt, die sind da, die Standards. Im Prinzip ist es aber so, dass wir die Standards und Technologien entsprechend da haben, wo halt auch die Produkte mit drin sind, verwendet werden. Und wir orientieren uns gnadenlos auch an der föderalen Architekturrichtlinie, die es gibt für die Bundesrepublik. Das ist sozusagen das übergreifende Dokument, das wir haben. Woraus wir dann entsprechend alle anderen Sachen ableiten.
 \\  \noindent\rule{\textwidth}{0.4pt}\\  \textbf{Interviewer:} Also es gibt eine Architekturrichtlinie für Infrastruktur vom Bund?
 \\  \noindent\rule{\textwidth}{0.4pt} \\  \textbf{SA1:} Ja, nicht nur für Infrastruktur, für alles. Es gab früher eine Architekturrichtlinie des Bundes, die war zuständig oder verbindlich für alle Bundesbehörden. Die ist letztes Jahr erweitert worden zur föderalen Architekturrichtlinie. Die gilt jetzt sowohl für die Behörden im Bundesumfeld, als aber auch für Kommunalen und Länder runter.
 \\  \noindent\rule{\textwidth}{0.4pt}\\  \textbf{Interviewer:} Ist da auch die Modellierungssprache?
 \\  \noindent\rule{\textwidth}{0.4pt} \\  \textbf{SA1:} Da sind auch Modellierungssprachen drin gesetzt und vorgegeben.
 \\  \noindent\rule{\textwidth}{0.4pt}\\  \textbf{Interviewer:} Okay, also ihr hattet quasi auch die Vorgabe, ihr müsst in Richtung was allgemeinbekanntem gehen wie UML und nicht in irgendwelche Sonderlinge reingehen.
 \\  \noindent\rule{\textwidth}{0.4pt} \\  \textbf{SA1:} Ja, das dient die Architekturrichtlinie und dann früher die Architekturrichtlinie des Bundes dient dazu, dass auch die Bundesministerien und Behörden
untereinander einigermaßen ähnlich sind, die sich austauschen können. Und das ist jetzt mit der Föderalen erweitert worden, auch die Bundesländer und die Kommunen und so weiter.
 \\  \noindent\rule{\textwidth}{0.4pt}\\  \textbf{Interviewer:} Dann habe ich noch eine Frage zu deinen Skills. Wie hältst du dich up to date in dieser Sparte?
 \\  \noindent\rule{\textwidth}{0.4pt} \\  \textbf{SA1:} Das ist eine gute Frage. Ich versuche, soweit es geht, mich regelmäßig über Neuerungen zu informieren, wenn es geht. Und ich versuche auch, regelmäßig Fingerübungen zu machen, um in den Sachen drin zu bleiben. Dass ich einfach Modelle mache. Was ich jetzt zum Beispiel unter anderem mache, ich mache jetzt gerade die IT-Architekten-Qualifikation bei uns, die ich begleite. Und da gibt es auch ein Modul BPMN, UML und so weiter. Und dass ich da aktiv mitgestalte und da auch entsprechend die Beispiele mitentwickle, die die Teilnehmer sozusagen machen müssen, machen sollen. Und dadurch versuche ich, mich entsprechend up to date zu halten.
 \\  \noindent\rule{\textwidth}{0.4pt}\\  \textbf{Interviewer:} Also am meisten alles im Arbeitsumfeld? Oder sagst du auch nach der Arbeit, ich höre mir jetzt mal, ich sage jetzt mal das bekannteste Softwarearchitektur im Stream an oder so irgendwas?
 \\  \noindent\rule{\textwidth}{0.4pt} \\  \textbf{SA1:} Muss ich zugeben, ich mache zwar viel außerhalb der Arbeit, aber das nicht.
 \\  \noindent\rule{\textwidth}{0.4pt}\\  \textbf{Interviewer:} Okay. Dann bedanke ich mich nochmal für die Zeit.
 \\  \noindent\rule{\textwidth}{0.4pt} \\  \textbf{SA1:} Gerne.
 \\  \noindent\rule{\textwidth}{0.4pt}\\  \textbf{Interviewer:} ich hoffe, dass alles mit den Aufnahmegeräten funktioniert hat, aber es sieht von meiner Seite aus gut aus.
 \\  \noindent\rule{\textwidth}{0.4pt} \\  \textbf{SA1:} Bei Rückfragen melde ich mich nochmal.
 \\  \noindent\rule{\textwidth}{0.4pt}\\  \textbf{Interviewer:} Wenn du prinzipiell Interesse hast, mal die Auswertung zu sehen, werde ich dir dann die Masterarbeit, wenn sie fertig ist und bestanden, leite ich sie weiter. Das wird allerdings noch, also fertig muss sie am 2. März sein. Die Beurteilung dauert etwa zwei Monate
 \\  \noindent\rule{\textwidth}{0.4pt} \\  \textbf{SA1:} Ich weiß, ich habe gerade eine andere Masterarbeit, die ich auch Gutachte. Da ist jetzt nächste Woche die Mündlich-Verteidigung.
 \\  \noindent\rule{\textwidth}{0.4pt}\\  \textbf{Interviewer:} Dann, ja, wie gesagt, du bist wieder ein freier Mensch. Warst du ja vorher schon. Du kannst ganz gerne aus dem Raum wieder rausgehen, wenn du möchtest.
 \\  \noindent\rule{\textwidth}{0.4pt} \\  \textbf{SA1:} Na klar, mache ich. Danke dir. Viel Erfolg für deine Arbeit.
 \\  \noindent\rule{\textwidth}{0.4pt}\\  \textbf{Interviewer:} Dankeschön. Tschüss.
 \\  \noindent\rule{\textwidth}{0.4pt} \\  \textbf{SA1:} Okay, tschüss.
 \\  \noindent\rule{\textwidth}{0.4pt}\\  \textbf{Interviewer:} Tschüss.


\section{Interview 2}

\section{Interview 3}
\section{Interview 4}
\section{Interview 5}
\section{Interview 6}
\section{Interview 7}
\section{Interview 8}
\section{Interview 9}
\section{Interview 10}


